%%% --- Initialising document and class --- %%%
\documentclass[
    paper=a4, 
    10pt,
    headings=small
]{scrartcl} % report more appropriate, allows abstract

%%% --- Packages to load --- %%%

\usepackage{CormorantGaramond}
\usepackage[T1]{fontenc}
% \usepackage[light, condensed]{roboto}
%%% Using Robot sans serif font from https://tug.org/FontCatalogue/
%\usepackage{crimson}
%%% Using Crimson Text font from https://tug.org/FontCatalogue/crimsontext/
%%% NB this only changes the serif font

\addtokomafont{disposition}{\rmfamily\scshape}
% changes sections from the ugly koma bold sans to elegant small caps
\addtokomafont{descriptionlabel}{\rmfamily\scshape}
% does the same for the labels in the 'description' environment

\usepackage[
% colorlinks=true 
hidelinks
]{hyperref}
%%% Allows hyperlinks for all cross-references. 'hidelinks' turns off the off-putting colour boxes 
%% If i want the text to be coloured, but to make this attractive I need to use the xcolor package and adjust them otherwise they'll be red/lime green :-/

\usepackage{siunitx}
%%% to allow nice presentation of units 
\sisetup{detect-all}
%needed to make SI units use text font rather than Maths font, which just looks weird in a body of text to me.

\usepackage{booktabs}
%%% high quality tables

\usepackage{setspace}
%\doublespacing
\onehalfspacing
%%% sets onehalf spacing as per university requirements

\usepackage[version=4]{mhchem}
%%% Allows easy typesetting of chemical formulae

\usepackage{comment}
%%% To allow the comment environment for editing or debugging

\usepackage{graphicx}
%%% To allow import of images

\usepackage{subcaption}
%% Allows subfloats, ie multipicture figures. Can have captions for each subfloat plus a global caption.

\usepackage[
final,
tracking=smallcaps,
expansion=alltext,
protrusion=true,
]{microtype}


%%% from page 52 of LaTeX and friends - does better kerning and smallcaps spacing

%\usepackage{chngcntr}
%\counterwithout{figure}{chapter}
%\counterwithout{table}{chapter}
% allows figures and tables to count consecutively rather than restarting every chapter.
% was useful for ESA but probably not for thesis.

%\usepackage{makeidx}
%\makeindex
%%% Allows generation of an index using the \index{key} and \printindex commands and the `makeindex foo.idx` command

%\usepackage{showidx}
%%% a proofing tool that shows the index entries in the margin. turn off for final version/easier reading

%%% --- Bibliography using biblatex --- %%%
\usepackage[
backend=biber,
style=authoryear,
citestyle=authoryear
%style=chem-acs,
%citestyle=chem-acs,
]{biblatex}

 \usepackage{todonotes}
% Allows todo notes scattered throughout, with a todolist automatically generated at the start.
% can be slow when compiling a big doc so might be worth changing later on?

\usepackage[noabbrev]{cleveref}
% An automated referencing package, looks good.

\usepackage[acronym,toc,nogroupskip]{glossaries}
\setlength{\glslistdottedwidth}{0.3\textwidth}

%\usepackage{minted}
% package allows typesetting and import of code
   %% Commented out until needed to make pdflatex run without shell escape
%\usepackage{pgfgantt}
%\usepackage{lscape}
%\usetikzlibrary{shapes,arrows}
%From irp to allow gantt chart

\addbibresource{/Users/jim/Library/CloudStorage/Box-Box/PhD/thesis/support/references.bib}
\let\cite\parencite
\renewcommand*{\bibfont}{\small}

\usepackage{pdflscape}

% \setlength{\parskip}{1em} % Changes paragraph formatting so each starts after a space
