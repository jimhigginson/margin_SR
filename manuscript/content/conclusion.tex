Here's where I use Hutan's suggestion to put something catchy and sexy.

For decades, the only adjunct to surgeon preference in head and neck surgical oncology has been frozen section.
Our meta-analysis shows that in controlled studies of diagnostic accuracy, frozen section performs well, however in clinical practice the limitations of economic cost, sampling error and addition to an already lengthy surgical time means that they are not routinely used despite the significant consequences of positive margins for patients and healthcare systems.

However, our study has systematically reviewed and analysed the diagnostic accuracy of newer techniques, and in particular, tissue targeted fluorescence studies using fluorescent-conjugated antibodies or pH-activated nanoprobes appear to hold significant potential as intraoperative diagnostic margin assessment techniques.
The studies included in this meta-analysis are all phase I trials, but further phase II/III trials are ongoing (LOOK THESE UP), and the head and neck surgical oncology community can await the results with fervent anticipation.

(perhaps this new information should go into the discussion?)
