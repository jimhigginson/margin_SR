\subsection{Study Selection}

The electronic literature searches and careful bibliographic examination identified a total of (X) citations for initial screening.
The PRISMA diagram is shown in \cref{fig:prisma} on \cpageref{fig:prisma}.
Following removal of duplicate citations, (X) unique citations remained.
(X) of these were excluded based on the title and abstract.
Of the remaining (X) studies that underwent full text screening, (X) were excluded.
Papers were excluded if (list reasons here.)

A total of (X) papers fulfilled the inclusion criteria and were included for systematic review and meta-analysis.
In (X) papers, diagnostic accuracy data were directly provided. 
In (X) papers, diagnostic accuracy data was calculated from the derived diagnostic metrics provided.

Include the numbers of studies in each overall category here.
Quality assessment outcomes using the QUADAS-2 and SORT systems are shown in table X.




\subsection{Study Characteristics}

\subsection{Meta-analysis}

\paragraph{Frozen section}

14 studies reported diagnostic accuracy data for frozen section, including one study \cite{amitImprovingRateNegative2016} that reported 2 cohorts, one with defect driven frozen section analysis, and one with specimen-drive analysis.
Pooled sensitivity was 0.795 (95\% CI 0.668--0.882); pooled specificity was 0.991 (0.979--0.996).
The diagnostic odds ration was 309.8 (153.5--625.6).
Heterogeneity for univariate analysis (DOR) was substanital, with Higgins $I^2 = 68\%$, $(p<0.01)$.
The bivariate \gls{sroc} curve is shown in \cref{fig:frozen_sroc} on \cpageref{fig:frozen_sroc}, and the pooled weighted \gls{auc} was 0.976. 
\todo{note 95\% CI not availabel for AUC in the MADA package}.

\paragraph{Tissue-specific Fluorescence}

5 studies presented diagnostic accuracy data on tissue-targeted fluorescence techniques.
Pooled sensitivity (95\% confidence interval) was 0.957 (0.839--0.990); pooled specificity was 0.827 (0.747--0.886).
The diagnostic odds ratio was 66.4 (37.2--118.7)
These studies had a very low level of heterogeneity, with Higgins $I^2 = 0\%$, $(p=0.58)$.
The bivariate \gls{sroc} curve is shown in \cref{fig:chemo_sroc} on \cpageref{fig:chemo_sroc}, and the pooled weighted \gls{auc} was 0.944.

\paragraph{Optical Techniques}

10 studies reported diagnostic accuracy data for techniques using optical evaluation of tissue.
Pooled sensitivity was 0.919 (95\% CI 0.861--0.954); pooled specificity was 0.855 (0.707--0.935).
The diagnostic odds ratio was 58.9 (21.3--163.1).
There was substantial heterogeneity: the Higgins $I^2$ for the univariate (DOR) analysis was 75\%, $(p<0.01)$.
The bivariate \gls{sroc} curve is shown in \cref{fig:optical_sroc} on \cpageref{fig:optical_sroc}, and the pooled weighted \gls{auc} was 0.925.

\paragraph{Touch Imprint Cytology}

3 studies presented diagnostic accuracy data on the intraoperative use of touch imprint cytology.
Pooled sensitivity was 0.925, with 95\% confidence intervals of 0.351--0.996; pooled specificity was 0.988 with 95\% confidence interval of 0.212--1.00.
The diagnostic odds ratio was 51.1 (95\% CI 12.7--205.4).
There was a low level of between study heterogeneity, higgins $I^2 = 22\%$, $(p=0.28)$.
The bivariate \gls{sroc} curve is shown in \cref{fig:tic_sroc} on \cpageref{fig:tic_sroc}.The pooled, weighted \gls{auc} was 0.919.

\paragraph{Topical Staining}

4 studies reported diagnostic accuracy data for direct staining of tissue.
Pooled sensitivity was 0.918 (95\% Confidence Intervals 0.770--0.974); pooled specificity was 0.759 (0.414--0.953).
The diagnostic odds ratio was 16.4 (4.3--62.1).
Heterogeneity for the univariate diagnostic odds rato was moderate with Higgins $I^2 = 43\%$, $(p=0.15)$.
The bivariate \gls{sroc} curve is shown in \cref{fig:stain_sroc} on \cpageref{fig:stain_sroc}, with a pooled weighted \gls{auc} of 0.883.



Risk of bias within each study
Results of individual studies
Synthesis of results
Risk of bias across studies
Additional analysis

\subsection{Publication Bias}

The funnel plot confirms the findings of the forest plot, showing that there is a considerable amount of heterogeneity in the studies published.
Visual inspection of the plot does not show any drastic publication bias.

The best numerical quantification of publication bias is the \emph{trim and fill} method \cite{duvalTrimFillSimple2000}
This method evaluates the distirbution of the univariate model x precision space.
The best way is plotting diagnostic odds ratio against standard error.
The trim and fill model then evaluates the symmetry of this plot, and estimates the position of studies that would be 'missing' from this plot in a truly representative meta-analysis.
The number of 'missing' studies, $K_0$, acts as a quantification of publication bias.
A meta-analysis with a trim and fill $K_0 = 0$ would have no publication bias.
The more studies are reuqired to balance the distribution, the higher the publication bias.
This method has been shown to be the simplest and most representative method for quantification of publication bias \cite{burknerTestingPublicationBias2014}.


This analysis shows $K_0 = 3$, indicating a relatively low level of publication bias.

The reconstructed/trim-filled funnel plot (\cref{fig:funnels} on \cpageref{fig:funnels}) shows the estimates of the missing studies as showing low treatment effect with low precision.
This fits the expectation of publication bias, where smaller studies that do not show a significant, novel or contradictory effect are less likely to be submitted or, if submitted, accepted for publication.
