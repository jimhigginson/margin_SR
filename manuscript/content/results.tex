\subsection{Study Selection}

The electronic literature searches and careful bibliographic examination identified a total of 2344 citations for initial screening.
Following removal of duplicate citations, 2082 unique citations remained.
1911 of these were excluded based on the title and abstract.
Of the remaining 169 studies that underwent full text screening, 134 were excluded.
The most common reasons for exclusion were because raw diagnostic data were not presented or calculable from data presented in the manuscript (57), or because the reference identified in the search was an abstract only, not a full peer-reviewed paper (45).
The PRISMA diagram is shown in \cref{fig:prisma} on \cpageref{fig:prisma}.

A total of 35 papers fulfilled the inclusion criteria and were included for systematic review.
One paper \cite{amitImprovingRateNegative2016} compared two separate methods for use of frozen section, and so 36 sets of diagnostic data were included in the final meta-analysis.
\footnote{In (X) papers, diagnostic accuracy data were directly provided, In (X) papers, diagnostic accuracy data was calculated from the derived diagnostic metrics provided.}

13 papers contributed 14 datasets to the frozen section category for meta-analysis.
10 papers presented diagnostic accuracy data for optical techiniques.
This was the most heterogeneous group, with 8 different techniques presented.\footnote{should I enumerate them here? They're spelled out in the discussion}
5 papers reported diagnostic data for tissue targeted fluorescence techniques.
Topical staining techniques were reported in 4 studies, and touch imprint cytology included 3 papers.

\subsection{Study Characteristics}
27 papers used a prospective methodology, 9 were retrospective.
All of the retrospective papers reported frozen section results.
The publication dates spanned almost 50 years, from 1973 to 2021, though 26 of the 35 studies were published in or after 2016.
The mean or median age was presented in X studies, with a range from Y to Z.
Quality assessment outcomes using the \gls{quadas2} and \gls{sort} scoring systems are shown in \cref{tab:qual_scores} on \cpageref{tab:qual_scores}.\footnote{summarise the quality scores}






\subsection{Meta-analysis}

\paragraph{Frozen section}

14 studies reported diagnostic accuracy data for frozen section, including one study \cite{amitImprovingRateNegative2016} that reported 2 cohorts, one with defect driven frozen section analysis, and one with specimen-drive analysis.
Pooled sensitivity was 0.795 (95\% Confidence Intervals (CI) 0.668--0.882); pooled specificity was 0.991 (0.979--0.996).
The \gls{dor} was 309.8 (153.5--625.6).
Heterogeneity for univariate analysis (\gls{dor}) was substantial, with Higgins $I^2 = 68\%$, $(p<0.01)$.
The bivariate \gls{sroc} curve is shown in \cref{fig:frozen_sroc} on \cpageref{fig:frozen_sroc}, and the pooled weighted \gls{auc} was 0.976. 

\paragraph{Tissue-specific Fluorescence}

5 studies presented diagnostic accuracy data on tissue-targeted fluorescence techniques.
Pooled sensitivity (95\% confidence interval) was 0.957 (0.839--0.990); pooled specificity was 0.827 (0.747--0.886).
The \gls{dor} was 66.4 (37.2--118.7)
These studies had a very low level of heterogeneity, with Higgins $I^2 = 0\%$, $(p=0.58)$.
The bivariate \gls{sroc} curve is shown in \cref{fig:chemo_sroc} on \cpageref{fig:chemo_sroc}, and the pooled weighted \gls{auc} was 0.944.

\paragraph{Optical Techniques}

10 studies reported diagnostic accuracy data for techniques using optical evaluation of tissue.
Pooled sensitivity was 0.919 (95\% CI 0.861--0.954); pooled specificity was 0.855 (0.707--0.935).
The \gls{dor} was 58.9 (21.3--163.1).
There was substantial heterogeneity: the Higgins $I^2$ for the univariate (DOR) analysis was 75\%, $(p<0.01)$.
The bivariate \gls{sroc} curve is shown in \cref{fig:optical_sroc} on \cpageref{fig:optical_sroc}, and the pooled weighted \gls{auc} was 0.925.

\paragraph{Touch Imprint Cytology}

3 studies presented diagnostic accuracy data on the intraoperative use of touch imprint cytology.
Pooled sensitivity was 0.925 (95\% CI 0.351--0.996); pooled specificity 0.988 (0.212--1.00).
The \gls{dor} was 51.1 (12.7--205.4).
There was a low level of between study heterogeneity, higgins $I^2 = 22\%$, $(p=0.28)$.
The bivariate \gls{sroc} curve is shown in \cref{fig:tic_sroc} on \cpageref{fig:tic_sroc}.The pooled, weighted \gls{auc} was 0.919.

\paragraph{Topical Staining}

4 studies reported diagnostic accuracy data for direct staining of tissue.
Pooled sensitivity was 0.918 (95\% CI 0.770--0.974); pooled specificity was 0.759 (0.414--0.933).
The diagnostic odds ratio was 16.4 (4.3--62.1).
Heterogeneity for the univariate diagnostic odds rato was moderate with Higgins $I^2 = 43\%$, $(p=0.15)$.
The bivariate \gls{sroc} curve is shown in \cref{fig:stain_sroc} on \cpageref{fig:stain_sroc}, with a pooled weighted \gls{auc} of 0.883.


\subsection{Publication Bias}

The funnel plot confirms the findings of the forest plot, showing that there is a considerable amount of heterogeneity in the studies published.
Visual inspection of the plot does not show any drastic publication bias.

The best numerical quantification of publication bias is the \emph{trim and fill} method \cite{duvalTrimFillSimple2000}
This method evaluates the distirbution of the univariate model x precision space.
The best way is plotting diagnostic odds ratio against standard error.
The trim and fill model then evaluates the symmetry of this plot, and estimates the position of studies that would be 'missing' from this plot in a truly representative meta-analysis.
The number of 'missing' studies, $K_0$, acts as a quantification of publication bias.
A meta-analysis with a trim and fill $K_0 = 0$ would have no publication bias.
The more studies are reuqired to balance the distribution, the higher the publication bias.
This method has been shown to be the simplest and most representative method for quantification of publication bias \cite{burknerTestingPublicationBias2014}.


This analysis shows $K_0 = 3$, indicating a relatively low level of publication bias.

The reconstructed/trim-filled funnel plot (\cref{fig:funnels} on \cpageref{fig:funnels}) shows the estimates of the missing studies as showing low treatment effect with low precision.
This fits the expectation of publication bias, where smaller studies that do not show a significant, novel or contradictory effect are less likely to be submitted or, if submitted, accepted for publication.
