%Structured based on the PRISMA reporting guidelines
% Introductory paragraph should include  

This systematic review was conducted in accordance with the guidelines for ``Preferred Reporting Items for Systematic Reviews and Meta-Analyses'' (PRISMA). %\cite{moherPreferredReportingItems2010}


%Protocol and registration

Write a protocol for Prospero??

The meta-analysis was registered on Prospero before starting the searches. It was assigned the registration number CRD42021287618.

\subsection{Literature Search}

To identify relevant studies, bibliographic searches were performed in Medline, Embase, Cochrane Library and Scopus databases.

Relevant studies were identified using electronic bibliographic searches (up to date as of January 2016) in PubMed, Cochrane Library, Scopus, and EMBASE. 
MESH terms and all-field search terms were searched for ``Head and Neck Squamous Cell Carcinoma'' [head and neck, pharyn*, hypopharyn* oropharyn*, laryn*, oral, tongue][squamous cell carcinoma, carcinoma, cancer, tumour, tumor] AND ``margin'' OR ``margin assessment'' or ``margins of excision'' [MESH] %Check the mesh headings to include here - do they match the terms I used? 
The search included all study designs. 
Potentially relevant studies identified from bibliographies of manuscripts included in the full text screen were added to the database of studies to screen. 
Individual searches were also performed in Google Scholar as suggested by full text screening no further potentially relevant studies were identified for screening. 
No limitations were placed on the search date range. 
The last search was performed in October 2021.


Two investigators (JAH and OB) screened all titles and abstracts independently. Any title or abstract identified as potentially relevant by either investigator was included for full-text review. 


\subsection{Inclusion Criteria}

Any study that included diagnostic margin assessment data provided by one or more intraoperative techniques used during curative surgery for HNSCC in adults (mean age \textgreater 18) was eligible for inclusion.
Studies were only included for meta-analysis if they provided raw diagnostic data (true positive, false positive, true negative, false negative) for the index test compared with permanent section histopathology or sufficient derived diagnostic metric data (total sample numbers, sensitivity, specificity, PPV and NPV) to reconstruct the raw diagnostic data.

\subsection(Exclusion Criteria}

Abstracts, conference proceedings, opinion articles, case reports, reviews, meta-analyses were excluded, as were articles that were not available in English.
Studies not performed in humans or not in an operative environment were excluded. 
Studies were excluded if they did not report raw diagnostic data, or sufficient derived diagnostic metric data 

Disagreement regarding article inclusion was resolved by consensus.

\subsection{Study Quality}

Each study was assessed for methodological quality using two separate validated scoring systems.

The Quality Assessment of Diagnostic Accuracy Studies 2 (QUADAS--2) checklist was used to provide an transparent rating of bias and applicability \cite{whitingQUADAS2RevisedTool2011}, with a score out of 14 assigned to each primary diagnostic study.

The Strength Of Recommendation Taxonomy (SORT) numerical scale for diagnostic studies scored out of 3.34 
\cite{ebellStrengthRecommendationTaxonomy2004}

% I think I need to redo the QUADAS and SORT scoring for all of them :-/


Both checklists were independently completed by the 2 investigators (E.R.S and R.A.) with agreement and therefore subsequently used to rate all included studies. 

All 7 QUADAS-2 questions were considered relevant for inclusion. 
For each of the seven QUADAS-2 questions, the study was assigned a score of 1 if there was a low risk of bias, and 2 if the risk was high or unclear.

To consider the conduct or interpretation of the index test adequate, we required study authors to report the number of margins obtained per specimen, as well as provide information on the embedding, sectioning, and staining of the derived specimen. 

The specialty of the operator (ie, pathologist,surgeon, or radiologist) interpreting the results had to be stated. 

We defined ‘‘bias’’ of the reference standard as occurring when the precise width (mm) between tumor and healthy tissue defining a ‘‘positive margin’’ was not reported. 

We determined a ‘‘bias’’ in flow and timing to have occurred when the number of patients selected for inclusion did not correspond to the number of patients included in the statistical analysis, with the exception of reports for which all patients were accounted for despite not being included in outcome data. 

The SORT scoring system was used to assess the diagnostic quality of the studies. 
High quality prospective trials or cohort studies were assigned a score of 1, lower quality cohort studies or diagnostic case-control studies were assigned a score of 2.

\subsection{Data Collection}

All data were extracted using Covidence \cite{Covidence}



\subsection{Meta-analysis}

Results were synthesised using the meta and mada packages in R. 
