\subsection{Summary of findings}

This study haseperformed a systematic reviem and meta-analysis of the diagnostic accuracy of established andemerging intraoperative margin assessment techniques.
The results we provide act as a benchmark for comparison with emerging techniques.
The rising incidence of HNSCC, driven by HPV, and the clear desire from patient preference data to avoid recurrence and morbidity associated with adjuvant treatment means that there is a pressing need to develop IMA techniques with satisfactory metrics for widespread adoption into surgical workflow.

Our data suggests that the two best IMA techniques are frozen section (insert sensitivity and specificity here), and chemiluminescent techniques (insert pooled sensitivity and specificity here), with an AUC of ($x$ \& $y$) respectively.
Despite these promising results, neither technique is widely used in current clinical practice
\emph{summary of the reasons for each here before going into detail.}

\subsection{Frozen section}

\subsection{Chemiluminescence}



\subsection{Relevance of findings to clinicians, researchers and policy-makers}



\subsection{Limitations}

High levels of heterogeneity.
Papers variably reporting per sample or per patient diagnostic accuracy.

In order to meaningfully compare different intraoperative diagnostic modalities, we have elected to group together related but heterogeneous IMA techniques. 
This heterogeneity limits the generalisability of the results.
The 'optical' category is the most heterogeneous, which corresponds to the novelty of this area of research, and the multiple approaches taken by researchers to exploit the differential optical properties of malignant tissue for rapid diagnostics.

Optical techniques as grouped here all measure the signal returned when light is shone on the target tissue.
Autofluorescence with the Velscope measures the photons emitted by fluorescence when monochromatic light is shone on the tissue.
Optical coherence tomography measures the refractive properties of tissue to construct a composite image of the tissue in 3D.
Raman spectroscopy quantifies the inelastic scattering of photons from tissue illuminated with monochromatic light.


\subsection{The role of IMAs}

A question that remains unanswered is whether the use of IMAs to achieve clear margins results in improved oncological outcomes (LOOK DEEPER INTO THIS AND CITE).
The underlying assumption of IMAs is that association between positive surgical margins and poor oncological outcomes is \emph{causative}.
This reasoning assumes that positive margins imply viable malignant cells left \emph{in situ}, with the ability to continue multiplication, local invasion and ultimately acquisition of the mutations necessary to metastasise to distant sites.
However, it is possible that resection with positive margins---where this was not anticipated by the surgeon---reflects a fundamentally aggressive tumour phenotype that is likely to recur whether or not IMAs facilitate resection with clear margins.
There is some evidence to support this theory.
Where multipleresectionsare required to achieve clear margins,disease-free and overall-survival is significantly worse \cite{mooreTransoralRoboticSurgery2018}
