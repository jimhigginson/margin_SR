Rough table of contents for discussion:
\begin{itemize}
\item Summary of findings
\item Frozen section
\item Targeted fluorescence
\item Optical techniques
\item The rest
\item Diagnostic accuracy =/= improved outcomes
\item Significance of these results
\item Limitations
\end{itemize}
 
\subsection{Summary of findings}

This study has performed a systematic review and meta-analysis of the diagnostic accuracy of established and emerging intraoperative margin assessment techniques.
The results we provide act as a benchmark for comparison with emerging techniques.
The rising incidence of \gls{hnscc}, driven by \gls{hpv}, and the clear desire from patient preference data to avoid recurrence and morbidity associated with adjuvant treatment means that there is a pressing need to develop \gls{ima} techniques with satisfactory metrics for widespread adoption into surgical workflow.

Here is where Omar suggests that I introduce the concept of sensitivity and specificity for greater clarity in the subsequent discussions


Standardised metrics of diagnostic performance are important for fair comparison of diagnostic tests.
Sensitivity and specificity are important metrics for understanding the performance of a diagnostic test.
They are widely used, and useful.
Sensitivitiy measures the rate at which the presence of disease is accurately diagnosed. 
A sensitivity of \SI{100}{\percent} implies that the disease will always correctly identify a patient who has the disease.
Conversely, specificity measures the rate at which the \emph{absence} of disease is correctly diagnosed. 
A test with \SI{100}{\percent} specificity would never identify a healthy patient as having the disease.
\footnote{should I create a sidebox/table showing the formulae for sens/spec/fpr/fnr/accuracy?}

In practical usage, no tests have \SI{100}{\percent} specificity and sensitivity.
Adjusting the diagnostic threshold to improve one frequently worsens the other.
This relationship is demonstrated in the \gls{roc} curve.
A test that performs better overall will have a \gls{roc} curve approaching the top left corner, with an \gls{auc} approaching 1.
A test that performs no better than random will have an \gls){auc} of $0.5$.

It is worth noting that, unlike false positive rate and false negative rate, sensitivity and specificity do not take into account the prevalance of disease. 
In the context of \gls{ima} techniques, the patient is known to have cancer, the question is whether the tissue sampled contains cancer or not. 
The rate at which these contain disease is highly variable, and depends on many complex factors, including the skill of the operating surgeon, the ability of the test to provide rapid results, and whether the disease has an aggressive, infiltrative pattern.
Evaluating sensitivity and specificity allows more standardised comparison.

When evaluating the diagnostic performance of \gls{ima} techniques, the purpose of the techniques should be borne in mind. 
The aim of \gls{ima} is to allow surgeons to remove disease entirely, to reduce risks of recurrence and the need for morbid adjuvant treatment.
Inappropriate removal of healthy tissue is to be avoided, but can be considered a secondary objective for \gls{ima} techniques.
As such, sensitivity is probably the most important metric to consider.

Our data suggests that the two best \gls{ima} techniques are frozen section, with pooled sensitivity \SI{79.5}{\percent} and specificity \SI{99.1}{\percent}, and tumour-targeted fluorescence techniques, with pooled sensitivity \SI{95.7}{\percent} and specificity \SI{82.7}{\percent}.
with an AUC of 0.976 and 0.944 respectively.
Despite these promising results, neither technique is widely used in current clinical practice
\emph{summary of the reasons for each here before going into detail.}
\footnote{put in a table outlining all the sens and specs for each modality and refer to it here}

\subsection{Frozen section}

Frozen section is a long-established adjunct for intra-operative diagnosis and margin evaluation.

In our meta-analysis, the specificity of frozen section was (\SI{99.1}{\percent}), the sensitivity was (\SI{79.5}{\percent}).
Despite these impressive numbers, however, frozen section has not become a routine part of head and neck resection surgery, and it is interesting to explore the reasons why.
A key reason for the limited uptake of frozen section is that the major source of error is not accounted for by sensitivity and specificity metrics.
For any given sample, errors due to technical processing and inaccurate interpretation are rare.
Instead, most errors occur at the sampling stage - when the surgeon selects the area of the tumour bed or resection specimen to sample\cite{layfieldFrozenSectionEvaluation2018}, which risks sampling healthy tissue adjacent to an inconspicuous nidus of residual disease \cite{due.RefiningUtilityRole2016}, or as a result of miscommunication between surgical and pathological teams \cite{blackc.CriticalEvaluationFrozen2006}
Tumour bed sampling is less accurate than a specimen driven approach \cite{maxwellEarlyOralTongue2015}.

Contributing to sampling error is the high logistical and economic cost of frozen section, that places a limitation on the volume of tissue that can be analysed.
Frozen section requires specialised equipment, and for an experienced histopathologist to be able to be available for \SIrange{20}{30}{\minutes} at short notice, with implications on their ability to do other routine pathological work for the health service. 
The economic costs of this are considerable: 
It is associated with significant cost, estimated at \$3123 per patient \cite{dinardoAccuracyUtilityCost2000}. 
(This is over \$5200 in today's money (https://www.usinflationcalculator.com))

In addition to the weaknesses of the technique in terms of diagnostic accuracy in the strict sense evaluated in this systematic review, there is a broader question of whether the use of frozen section, and a negative result provided by a frozen section, is able to add to the rates of local control following resection.

There is some evidence that the use of frozen section in transoral surgery reduces the rate of definitive positive margins (systematic review)\cite{gorpheSystematicReviewMetaanalysis2019}. This is also covered in the summary section below.


In some head and neck surgical contexts, the operation aims to remove the tumour with large margins - salvage laryngectomy and primary tongue resection are key examples. 
In this situation, the rates of positive margins are usually low, and so teh marginal benefit (no pun intended) of the use of frozen section is unlikely to be justified.
Note that \emph{all} of the studies that were retrospective were frozen section studies, which raises questions about the conclusions and comparisons that can be drawn from these data.

\subsection{Tissue-specific fluorescence}

Tumour-specific targeted fluorescence techniques were shown to have excellent diagnostic accuracy, with overall sensitivity of \SI{95.7}{\percent} and specificity of \SI{82.7}{\percent}. 
The \gls{auc} was 0.944.
These results suggest that targeted fluorescence techniques may have a major role in the future of head \& neck surgical oncology.

Tissue-specific fluorescence techniques use systemic administration of biocompatible fluorophores---most commonly indocyanine green---that concentrate in tumour tissue, allowing localisation of the tumour  by shining a light of appropriate wavelength on the area to elicit a fluorescent response.
A variety of different approaches were used to achieve preferential accumulation of fluorescent dye within tumour tissue.
Despite these variations, the concept appears to have a consistent benefit for detecetion of diseaose, with a pooled sensitivity of \SI{95.7}{\percent}, and resolution able in some cases to identify otherwise undetectable satellite metastases \cite{voskuilf.j.ImageguidedSurgeryTumor2019}.

There were two broad approaches in the included studies. 
The majority linked the fluorophore to an antibody targeted towards \gls{egfr}, which is overexpressed in \gls{hnscc}\footnote{CITATION}, either cetuximab or panitumumab.
This approach was the most studied, and has the advantage of exploiting exsisting treaotment options with xknown safety profiles.
Another advantage of this approach is that in could be combined with other labelling modalities, such as radio-labelling, allowing localisation of tumour deposits at depths greater than the 6 mm allowed by fluorescence.
A potential disadvantage  may be int setting where tumours, or subclonal 'parts' of tumours, do not express \gls{egfr}, ethier because they never did, or because it is downregulated following prior treatment.\footnote{find citations for the downregulation of egfr}

The alternative approach is selective, or conditional \textit{quenching}, whereby the fluorophore is administered enveloped in a nanoparticle micelle that inhibits fluorescence.
This micelle is stable until it encounters a pH below a tunable threshold, at which point the micelle irreversibly dissociates, releasing the fluorophore into the local tissue environment, leading to tissue-specific fluorescence.
This approach exploits the localised acidosis that is a near ubiquitous hallmark of solid cancers, and thus has the advantage of being 'tumour agnostic'.
However, unlike with tumour-targeted antibodies, this approach is less suited to coupling with radiolabelling approaches, as the quenching would have no effect on the transmission of (gamma - CHECK) radiation, so signal would not be confined to the tumour.

TODO - do I want to include a brief critique of this as not being suitable for 'in situ' open field work? 
THey've tried and it wasn't great and involved a lot of interruptions (See Gao 2018), and so now they take the specimen to an analysis box on a back table in theatre and analyse the specimen instead. 
Akin to specimen driven frozen section, so fine in that regard, but ultimately it is indirect information about the likelihood of viable tumour cells being left in situ.
A decent reference, in addition to Gao2018, for the discussion between open field and back table would be moore et al 2017 j nucl med, reference number 7 from the gao paper.

As is to be anticipated with an emerging technique, all (CHECK) of the studies of targeted fluorescence were pilot or phase I studies, with limited patient recruitment. However, the methodology and reporting was of a high standard.
Three of the papers (citations) dealt with ICG annealed to a HNSCC-specific antibody: either cetuximab or panitumumab.
Pros and cons of this: 80-90\% expression of EGFR
One paper  dealt with a different methods of activating the fluorophore in the tumour environment: a nano-polymer that quenches the ICG at physiological pH and unquenches in the acidic tumour microenvironment

It is notable that this work is mostly being carried out in the USA and the Netherlands.

A key feature of fluorophores that are activated by near infrared light is that the signal can be detected through tissue to a depth of around \SI{6}{\milli\meter}.
This 'shine through' allows non-invasive assessment of the margin clearance of specimens, as the clinically important margin cut-offs are 2 and 5 mm, although at shallower depths, specificity seems to suffer (van keulen).
There may be situations in which recognition of the presence of tumour at a deeper level would be advantageous, which this approach would not be able to do.
However, one possibility could be to conjugate the dye with a radioactive moiety, comparable to techniques employed in sentinel node biopsy. (CITE)

This could perhaps be overcome by the addition of a radioactive moiety, though this a) would add to the expense, b) would only be possible with tumour targeting approaches using EG antibodies, rather than tumour activated approaches, and c) comes with the potential risks associated with systemic administration of radioactive substances.

\subsection{Touch Imprint Cytology}

Other techniques within our meta-analysis were perhaps less compelling.

Touch imprint cytology only contributed three papers to the systematic review, and the results presented wide confidence intervals. 
It may have a role in resource constrained environments - as highlighted in one of the papers (WHICH) - as minimal complex equipment is required. 
However it is hard to see it having a role as the gold standard of care.

\subsection{Tissue staining techniques}

Application of dyes that differentially stain malignant and normal tissue have been known about for some time, but still are not in widespread use.
Lugol's iodine has been proposed as a useful adjunct for identifying dysplastic tissue from normal mucosa, by iodine staining starches present in normal tissue and absent in dysplastic tissue.
It has been investigated in a clinical trial, but the results have not yet been published (cite abstracts and clinical trial number).
Like all other stains, it is only of use at the start of the procedure, as it interacts unpredictably with the presence of blood in the operating field, making it impossible to accurately tumour from normal tissue at the deep margin consisting of dissected muscle and connective tissue.

Toluidine blue is a tissue dye - often used in examination of fixed tissue slides - that stains nucleic acids blue.
The high abundance of nucleic acids in rapidly multiplying malignant tissue means that it stains more intensely than adjacent normal tissue.
TODO---how is it at distinguishing malignancy and dysplasia? this should be a clear criticism. 
Also doesn't do teh deep margin

5-ALA (Leunig) and the gamma-gt spray (slooter) were both interesting - these stains were both stains that contained a fluorescent dye that was activated only in the presence of tumour. 
, or a precursor fluorophore that is activated in the presence of high concentrations of gamma-GT, an enzyme that is overexpressed in HNSCC (check this, slooter et al).

\subsection{Optical techniques}

Optical techniques as grouped here all measure the signal returned when light is shone on the target tissue.
Optical techniques showed some promise, but there was a significant amount of heterogeneity within and between studies. 
This is not surprising, as the group contained studies using a wide range of techniques, including (internet definitions) \ldots

\paragraph{High resolution microendoscopy} \cite{vilap.m.DiscriminationBenignNeoplastic2012, milesb.a.OperativeMarginControl2015}\footnote{define HRME}
\paragraph{Narrow band imaging} \cite{tirelliNarrowBandImaging2015, tirellig.TailoredResectionsOral2018}Narrowband imaging uses two specific wavelengths in the visible light spectrum to highlight the interpapillary capillary loop patterns in the mucosa, which are notably disorganised in tumour tissue \cite{vuEfficacyNarrowBand2014}.
\paragraph{Elastic scattering spectroscopy} \cite{grilloneg.a.ColorCancerMargin2017}
\paragraph{Coherent Raman scattering spectroscopy} \cite{hoeslir.c.CoherentRamanScattering2017} Raman spectroscopy quantifies the inelastic scattering of photons from tissue illuminated with monochromatic light.
\paragraph{Optomagnetic imaging spectroscopy} \cite{lisulb.PredictiveValueOptomagnetic2019}\footnote{define optomagnetic imaging spectroscopy} 
\paragraph{Multimodal nonlinear microscopy} \cite{heukes.MultimodalNonlinearMicroscopy2016}\footnote{define multimodal nonlinear microscopy---can this be bundled together with HRME?} 
\paragraph{Optical coherence tomography} (wikipedia) Optical coherence tomography (OCT) is an imaging technique that uses low-coherence light to capture micrometer-resolution, two- and three-dimensional images from within optical scattering media (e.g., biological tissue) \cite{hamdoonz.OpticalCoherenceTomography2016} \Gls{oct} uses changes in tissue optical properties---such as refractive index---to generate real-time, high-resolution cross-sectional images of tissue \cite{heidaria.e.UseOpticalCoherence2020}.
\paragraph{Tissue autofluorescence} \cite{ohnishiy.UsefulnessFluorescenceVisualization2016} Autofluorescence measures the photons emitted by fluorescence when monochromatic light is shone on mucosal tissue; a property which is lost by abnormal mucosa \cite{leey.-j.IntraoperativeFluorescenceGuidedSurgery2020}\footnote{check this citation}.

\subsection{Relevance of findings to clinicians, researchers and policy-makers}

Hutan made a compelling argument that the whole paper, but especially the discussion, should be compelling and exciting for clinicians in general and surgeons in particular if we are to get it published in a strong journal.

The full text screening stage of this systematic review was notable for the high number (check how many) of studies that did not provide raw diagnostic data, or a celar explanation of how diagnostic metrics were calculated, whether per patient, per sample or per test.
this represents a lost opportunity for systhesis of the hard work of colleoagus past \& present into our current understanding of intraoperative diagnsotics.

For this reason, along with previous authors \cite{stjohnDiagnosticAccuracyIntraoperative2017, irwigGuidelinesMetaanalysesEvaluating1994}, we call for all future publications of diagnostic studies in head and neck cancer to be contingent upon provision of raw diagnostic data---ideally in the form of a confusion matrix---and clear reporting of how the index and reference tests were performed.

Call for a clinical trial of targeted fluorescence?

\subsection{Limitations}

High levels of heterogeneity.
Papers variably reporting per sample or per patient diagnostic accuracy.

In order to meaningfully compare different intraoperative diagnostic modalities, we have elected to group together related but heterogeneous IMA techniques. 
This heterogeneity limits the generalisability of the results.
The 'optical' category is the most heterogeneous, which corresponds to the novelty of this area of research, and the multiple approaches taken by researchers to exploit the differential optical properties of malignant tissue for rapid diagnostics.


\subsection{The role of IMAs}

Improved diagnostic accuracy does not necessarily result in improved rates of negative margins, or indeed improved survival outcomes.
The limitations of frozen section illustrate this point clearly.
In UK practice, the NICE\footnote{acronymify} guidelines suggest frozen section be used where there is clinical suspicion of positive margins (cite).
In this context, the surgeon would usually take a sample from the wound bed.
The high specificity and sensitivity of frozen section means that it is likely to provide a correct diagnosis \textit{for that sample} unless there are technical failures during sample processing (CITE).
However, the success of this approach relies on the surgeon correctly identifying and sampling residual disease in the resection defect. 
If the surgeon inadvertently sample normal tissue adjacent to residual disease, the patient would have positive margins (with the attendent morbidity and mortality) despite frozen section correctly reporting a negative result.
This \textit{sampling error} is a key weakness of defect-drive frozen section (CITE), and is a key limitation to its wider adoption.

With this in mind, any new \gls{ima} technique should be able to provide results for the entiretyl of the specimen and/or the resection defect with high resolution whilst retaining excellent sensitivity and specificity.

This would be a good point to sing the praises of targeted fluorescence.
Also could float the idea of combining it, or some other targeted approach as a label for the iKnife.


Whatever the case is, ideally any and all new \gls{ima} techniques should be evaluated on their ability to reduce definitive positive margin rates, and to reduce rates of recurrence and disease specific morbidity.

Frozen section has been shown in a robust meta-analysis to reduce positive margin rates by 2--3 percentage points \cite{gorpheSystematicReviewMetaanalysis2019}.

However, it is possible that resection with positive margins---where this was not anticipated by the surgeon---reflects a fundamentally aggressive tumour phenotype that is likely to recur whether or not IMAs facilitate resection with clear margins.
There is some evidence to support this theory.
The revision of positive margins to negative guided by frozen section is a major risk factor for local recurrence \cite{ettlt.PositiveFrozenSection2016}, and where multiple resections are required to achieve clear margins, disease-free and overall-survival is significantly worse \cite{mooreTransoralRoboticSurgery2018}
14 to 22\% of patients may have a positive margin in the final specimen despite a negative intraoperative frozen section result \cite{ordAccuracyFrozenSections1997, due.RefiningUtilityRole2016}, probably representing sampling error.

Consider including unpublished iknife data? Perhaps the daoms abstract from Jag? Discuss with ZT.

