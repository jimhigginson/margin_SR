discussion goes here


\subsection{Publication Bias}

The funnel plot confirms the findings of the forest plot, showing that there is a considerable amount of heterogeneity in the studies published.
Visual inspection of the plot does not show any drastic publication bias.

The best numerical quantification of publication bias is the \emph{trim and fill} method (DUVAL\&TWEEDIE 2000).
This method evaluates the distirbution of the univariate model x precision space. 
The best way is plotting diagnostic odds ratio against standard error.
The trim and fill model then evaluates the symmetry of this plot, and estimates the position of studies that would be 'missing' from this plot in a truly representative meta-analysis.
The number of 'missing' studies, $K_0$, acts as a quantification of publication bias. 
A meta-analysis with a trim and fill $K_0 = 0$ would have no publication bias.
The more studies are reuqired to balance the distribution, the higher the publication bias.
This method has been shown to be the simplest and most representative method for quantification of publication bias (bu\"rkner and doebler 2014).


This analysis shows $K_0 = 3$, indicating a relatively low level of publication bias.

The reconstructed/trim-filled funnel plot (add and reference) shows the estimates of the missing studies as showing low treatment effect with low precision.
This fits the expectation of publication bias, where smaller studies that do not show a significant, novel or contradictory effect are less likely to be submitted or, if submitted, accepted for publication.
