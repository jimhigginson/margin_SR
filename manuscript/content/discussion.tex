
\subsection{Summary of findings}

This study has performed a systematic review and meta-analysis of the diagnostic accuracy of established and emerging intraoperative margin assessment techniques.
The results we provide act as a benchmark for comparison with emerging techniques.
The rising incidence of \gls{hnscc}, driven by \gls{hpv}, and the clear desire from patient preference data to avoid recurrence and morbidity associated with adjuvant treatment means that there is a pressing need to develop \gls{ima} techniques with satisfactory metrics for widespread adoption into surgical workflow.

Standardised metrics of diagnostic performance are important for fair comparison of diagnostic tests.
Sensitivity and specificity are important metrics for understanding the performance of a diagnostic test.
They are widely used, and useful.
Sensitivitiy measures the rate at which the presence of disease is accurately diagnosed. 
A sensitivity of \SI{100}{\percent} implies that the disease will always correctly identify a patient who has the disease.
Conversely, specificity measures the rate at which the \emph{absence} of disease is correctly diagnosed. 
A test with \SI{100}{\percent} specificity would never identify a healthy patient as having the disease.
\footnote{should I create a sidebox/table showing the formulae for sens/spec/fpr/fnr/accuracy?}

In practical usage, no tests have \SI{100}{\percent} specificity and sensitivity.
Adjusting the diagnostic threshold to improve one frequently worsens the other.
This relationship is demonstrated in the \gls{roc} curve.
A test that performs better overall will have a \gls{roc} curve approaching the top left corner, with an \gls{auc} approaching 1.
A test that performs no better than random will have an \gls{auc} of $0.5$.

It is worth noting that, unlike false positive rate and false negative rate, sensitivity and specificity do not take into account the prevalance of disease. 
In the context of \gls{ima} techniques, the patient is known to have cancer, the question is whether the tissue sampled contains cancer or not. 
The rate at which these contain disease is highly variable, and depends on many complex factors, including the skill of the operating surgeon, the ability of the test to provide rapid results, and whether the disease has an aggressive, infiltrative pattern.
Evaluating sensitivity and specificity allows more standardised comparison.

When evaluating the diagnostic performance of \gls{ima} techniques, the purpose of the techniques should be borne in mind. 
The aim of \gls{ima} is to allow surgeons to remove disease entirely, to reduce risks of recurrence and the need for morbid adjuvant treatment.
Inappropriate removal of healthy tissue is to be avoided, but can be considered a secondary objective for \gls{ima} techniques.
As such, sensitivity is probably the most important metric to consider.

Our data suggests that the two best \gls{ima} techniques are frozen section, with pooled sensitivity \SI{79.5}{\percent} and specificity \SI{99.1}{\percent}, and tumour-targeted fluorescence techniques, with pooled sensitivity \SI{95.7}{\percent} and specificity \SI{82.7}{\percent}.
with an AUC of 0.976 and 0.944 respectively.
Despite these promising results, neither technique is widely used in current clinical practice

\subsection{Frozen section}

Frozen section is a long-established adjunct for intra-operative diagnosis and margin evaluation.
In our meta-analysis, the specificity of frozen section was (\SI{99.1}{\percent}), the sensitivity was (\SI{79.5}{\percent}).
However, frozen section has not become a routine part of head and neck resection surgery.

A key reason for the limited uptake of frozen section is that the major source of error is not accounted for by sensitivity and specificity metrics.
For any given sample, errors of processing and interpretation are rare.
Instead, most errors occur at the sampling stage.
A common approach is for the surgeon to complete the tumour resection, then if concerns remain about risk of residual disease, they take samples from the tumour bed for frozen section analysis\cite{layfieldFrozenSectionEvaluation2018}.
This risks sampling healthy tissue adjacent to an inconspicuous nidus of residual disease \cite{due.RefiningUtilityRole2016}, or miscommunication between surgical and pathological teams \cite{blackc.CriticalEvaluationFrozen2006}.
These errors do not affect the sensitivity and specificity of frozen section, but markedly reduce its utility as an \gls{ima}.
Sampling errors can be reduced by taking a specimen-driven approach \cite{maxwellEarlyOralTongue2015}, where the whole surface of the specimen is sampled.
However, this substantially increases the time and cost per analysis.

Frozen section requires specialised equipment, and for an experienced histopathologist to be able to be available for \SIrange{20}{30}{\minute} at short notice, with implications on their ability to do other routine pathological work for the health service. 
The economic costs of this are considerable, estimated at \$3123 per patient \cite{dinardoAccuracyUtilityCost2000}, over \$5200 in today's money (https://www.usinflationcalculator.com).

\subsection{Tissue-specific fluorescence}

Tissue specific fluorescence is an exciting emerging technique that uses systemic administration of biocompatible fluorophores---most commonly indocyanine green---that preferentially accumulate in tumour tissue.
This allows tumour visualisation of the tumour by shining a light of appropriate wavelength on the area to elicit a fluorescent response.
In our meta-analysis, tumour-specific targeted fluorescence techniques had excellent diagnostic accuracy, with overall sensitivity of \SI{95.7}{\percent} and specificity of \SI{82.7}{\percent}. 
The \gls{auc} was 0.944.
As is to be anticipated with an emerging technique, all of the included studies were pilot or phase I studies, with limited patient recruitment. 
However, the methodology and reporting was of a high standard, with the work being carried out mostly in reputable units in the USA and the Netherlands.

In addition to the excellent sensitivity, this approach had resolution able in some cases to identify otherwise undetectable satellite metastases \cite{voskuilf.j.ImageguidedSurgeryTumor2019}, and allows detection of fluorescence signal through up to \SI{6}{\milli\metre} of tissue, potentially allowing estimation of the proximity of the tumour to the resection margin, albeit with reduced specificity \cite{vankeulenRapidNoninvasiveFluorescence2019}.
These results suggest that targeted fluorescence techniques may have a major role in the future of head \& neck surgical oncology.

Included in the meta-analysis were two different techniques for targeting the fluorescence within the tumour.
The majority linked the fluorophore to monoclonal antibodies  targeted at \gls{egfr}, which is almost always overexpressed in \gls{hnscc}\cite{chungIncreasedEpidermalGrowth2016}.
The fluorescent antibody binds to the surface of cells expressing \gls{egfr}, causing substantially increased fluorescent signal in tumour compared with adjacent normal tissue.

The other approach was conditional \textit{quenching}, whereby the fluorophore is enveloped in a nanoparticle micelle that inhibits fluorescence\cite{voskuilExploitingMetabolicAcidosis2020}.
This micelle is stable until it encounters a pH below a tunable threshold, at which point the micelle irreversibly dissociates. 
When administered intravenously, the micelle remains stable until it reaches the hypoxic, low-pH tumour microenvironment. 
Here, the micelle disociates, releasing the fluorophore, resulting in tissue-specific fluorescence.

Each of these two approaches have their advantages and disadvantages.
The monoclonal antibodies used in these studies (cetuximab \cite{warramFluorescenceImagingLocalize2016, voskuilf.j.FluorescenceguidedImagingResection2020}and panitumumab \cite{gaor.w.DeterminationTumorMargins2018, vankeulenRapidNoninvasiveFluorescence2019}) are already in clinical use, with a well established safety profile.
Repurposing already existing drugs provides greater confidence in the safety and side-effects profile, as well as speeding up development. 

Another advantage of using monoclonal antibodies, is that they could be combined with other labelling modalities.
One possibility is the use of radio-labelling, allowing localisation of tumour deposits at depths greater than the 6 mm allowed by fluorescence, comparable to the process of sentinel node biopsy.
However, this approach could not be used with nanoparticle micelles, as the quenching would have no effect on the transmission of gamma radiation, so signal would not be concentrated within the tumour.

The use of monoclonal antibodies targeting \gls{egfr} are limited bythe requirement for the tumour to express \gls{egfr}.
This is usually significantly upregulated in \gls{hnscc}, but can be suppressed in some tumour areas, particularly where the tumour microenvironment is hypoxic \cite{mayerDownregulationEGFRHypoxic2016}, a common feature in \gls{hnscc}\footnote{citation for hypoxia frequency in hnscc}.

The hypoxic tumour microenvironment and associated low pH is, however, the central premise behind the nanoprobe quenching approach to targeted fluorescence.
This has the further advantage of being `tumour agnostic'---applicable to any rapidly growing solid tumour.
Questions remain over whether it would be able to detect a relatively well vascularised satellite lesion distinct from the main tumour mass. 
Further work is needed to determine the optimal approach.

A common weakness of both methods tissue-targeted fluorescence as it is used in these studies is that it is difficult to use in situ as the operation is progressing. 
The signal \textit{in vivo} is \SI{87}{\percent} lower compared with imaging of the resected specimen, and repeated interruption to surgical workflow with the need to dim operative lights and pause to take fluorescence imaging made this approach inaccurate and impractical \cite{mooreCharacterizingUtilityLimitations2017}.
The authors now routinely take the resection specimen to an analysis box on a back table in theatre for analysis \cite{gaor.w.DeterminationTumorMargins2018}. 
This loses some of the immediacy of real-time feedback, but remains much faster than frozen section.

Read and cite this, it's really good\cite{leey.-j.IntraoperativeFluorescenceGuidedSurgery2020}\footnote{Work this citation in somewhere here, it's really good}

\subsection{Touch Imprint Cytology}

Touch imprint cytology involves pressing the surfaces of a completely resected specimen to glass slides.
This transfers cells from the specimen surface to the slide, allowing rapid evaluation for the presence of malignant cells on the cut surface of the specimen.
The pooled sensitivity was 0.925, and the specificity was 0.988, however there was significant heterogeneity within the three studies included,with a high degree of heterogeneity, and wide confidence intervals \cite{zafara.DiagnosticUtilityTouch2020, yadavg.s.IntraoperativeImprintEvaluation2013, naveedDiagnosticAccuracyTouch2017}.
The simplicity and speed of this technique is compelling, particularly resource constrained environments as minimal complex equipment is required \cite{naveedDiagnosticAccuracyTouch2017}.
However it is hard to see it having a role as the gold standard of care. Need to back this up with why.
All of the included studies were conducted in India or Pakistan.



\subsection{Tissue staining techniques}

Several techniques exploited the use of dyes or stains that differentially colour pathological and normal tissue when applied to the mucosa. 
When used intraoperatively, they can be used to highlight areas of disease that were not otherwise visually distinct from the adjacent normal tissue.
Four studies contributed to a pooled sensitivity of 0.918, with specificity 0.759.
Two studies evaluated stains which have been in clinical use for some time: Toluidine blue \cite{junaidm.ComparativeAnalysisToluidine2012} and Lugol's Iodine \cite{putriAceticAcidIodine2021}.
Toluidine blue is a tissue dye - often used in examination of fixed tissue slides - that stains nucleic acids blue.
The high abundance of nucleic acids in rapidly multiplying malignant tissue means that it stains more intensely than adjacent normal tissue.
Lugol's iodine stains starches that are present in normal tissue and absent in dysplastic or malignant tissue.
Although not universally adopted some units use it regularly. 
This systematic review highlights a need for further research into its diagnostic accuracy to support its continued use. % investigated in a clinical trial, but the results have not yet been published (cite abstracts and clinical trial number).
Neither of these techniques are able to distinguish between malignant and dysplastic tissue. 
This risks overtreatment as a result of false positives: dysplastic tissue being characterised as invasive cancer.

The other two studies both employed a fluorecent dye that was activated only in the presence of tumour.
One used a precursor fluorophore that is activated by \gls{ggt}, an enzyme that is overexpressed in \gls{hnscc} \cite{slooterm.d.DetectingTumourpositiveResection2018}.
The other used \gls{5ala}, which when applied to tissues lead to cellular occumulation fo protoporphyrin IX in tumour cells, which is highly fluorescent \cite{leunigFluorescenceStainingOral2001}.

A weakness of all staining techniques is that they are only of use at the start of the procedure.
They interact unpredictably with blood and other fluids in the operating field, making it difficult to accurately distinguish tumour from normal tissue at the deep margin consisting of dissected muscle and connective tissue.

\subsection{Optical techniques}

Optical techniques as grouped here all measure the signal returned when targeti tissue is illuminated with electromagnetic radiation, usually visible light.
In our meta-analysis they had a pooled sensitivity of 0.919, and a specificity of 0.855.
There was a significant amount of heterogeneity within and between studies. 
The studies included used a wide range of techniques.

Some optical techniques enhanced the visual contrast between diseased and healthy tissue during direct inspection of tissue by the surgeon. 
Narrow band imaging \cite{tirelliNarrowBandImaging2015, tirellig.TailoredResectionsOral2018} is a widely used diagnostic tool that uses dichromatic visible light to highlight interpapillary capillary loop patterns in the mucosa. 
These are notably disorganised in tumour tissue \cite{vuEfficacyNarrowBand2014}.
Tissue autofluorescence \cite{ohnishiy.UsefulnessFluorescenceVisualization2016} uses monochromatic light to activate fluorescence in normal tissue.
Through an optical filter, normal tissue appears bright green but abnormal mucosa loses the fluorescent properties and appears dark.
Both of these techniques are commonly used in the outpatient and pre-operative diagnostic settings, however there appears to be less support for their use as \gls{ima} devices, perhaps related to the presence of blood in an operative field. 
Also the restriction to evaluation of mucosa and not deep margins.
\Gls{oct} is an imaging technique that uses low-coherence light to capture micrometer-resolution, cross-sectional images of tissue \cite{hamdoonz.OpticalCoherenceTomography2016, heidaria.e.UseOpticalCoherence2020}, similar to the views provided by ultrasound imaging.

Other techniques offered ways to evaluated minimally processed tissue samples by evaluating the images or spectra returned from the tissue using a variety of modalities.
High resolution microendoscopy \cite{vilap.m.DiscriminationBenignNeoplastic2012, milesb.a.OperativeMarginControl2015} uses a fibre probe to perform in-theatre analysis of resected specimens.
The specimens are stained with proflavine as a nuclear stain, and the HRME probe was applied to the tissue, generating high-resolution, real time dynamic microscopic imaging of tissue architecture.
Pathological and normal tissue can be distinguished following minimal training.
Elastic scattering spectroscopy \cite{grilloneg.a.ColorCancerMargin2017}, coherent Raman scattering spectroscopy \cite{hoeslir.c.CoherentRamanScattering2017}, optomagnetic imaging spectroscopy \cite{lisulb.PredictiveValueOptomagnetic2019} and multimodal nonlinear microscopy \cite{heukes.MultimodalNonlinearMicroscopy2016} all used similar electromagnetic spectral analysis to evaluate ex vivo tissue samples.

Optical approaches show promise, but more research is needed before they can be considered a reliable \gls{ima}.

\subsection{The role of IMAs}
In this systematic review and meta-analysis, we have comprehensively and robustly evaluated the diagnostic accuracy of techniques used to identify tumour tissue at resection margins.
However, the metric of diagnostic accuracy should be considered in the broader context of clinical practice.
Improved diagnostic accuracy does not necessarily result in improved rates of negative margins, or indeed improved survival outcomes.
This is illustrated by the importance of sampling error in frozen section. 
14 to 22\% of patients may have a positive margin in the final specimen despite a negative intraoperative frozen section result \cite{ordAccuracyFrozenSections1997, due.RefiningUtilityRole2016}.
The high specificity and sensitivity of frozen section means that it is likely to provide a correct diagnosis \textit{for that sample} unless there are technical failures during sample processing (CITE).
Evaluating the faces of the specimen results in fewer sampling errors than a patient-driven approach \cite{aaboubouty.SpecimendrivenIntraoperativeAssessment2020, amitImprovingRateNegative2016, due.RefiningUtilityRole2016, maxwellEarlyOralTongue2015, bergeronm.DecreasingLocoregionalRecurrence2016}

With this in mind, any new \gls{ima} technique should be able to provide results for the entiretyl of the specimen and/or the resection defect with high resolution whilst retaining excellent sensitivity and specificity.
The results of this meta-analysis suggests that target fluorescence could present a powerful tool to overcome the risk of sampling error.
Targeted fluorescence provides a comprehensive view of the sample on a theatre back table, and provides spatially resolved data on the risk of positive or close margins.
In itself, this has excellent sensitivity and specificity. 
It may also work synergistically, either as a way to guide frozen section, or more ambitiously, as a complement to more advanced techniques.
Possibilities could include any of the techniques presented in this study, or could include technology such as the iKnife, which is able to provide near-instant tissue classification using surgical aerosol, either during dissection or directed sampling, and has shown great promise in early studies.


However, it is possible that resection with positive margins---where this was not anticipated by the surgeon---reflects a fundamentally aggressive tumour phenotype that is likely to recur whether or not IMAs facilitate resection with clear margins.
There is some evidence to support this theory.
The revision of positive margins to negative guided by frozen section is a major risk factor for local recurrence \cite{ettlt.PositiveFrozenSection2016}, and where multiple resections are required to achieve clear margins, disease-free and overall-survival is significantly worse \cite{mooreTransoralRoboticSurgery2018}

Clearly, these techniques require a great deal more research before they can be recommended as standard practice.

\subsection{Limitations}

The findings of this systematic review are limited by the heterogeneity of the included studies, particularly those in the optical and frozen section subgroups.
In addition to the variety of techniques included within these broader categories, there was considerable methodological heterogeneity too, with studies variably reporting per-sample or per-patient diagnostic accuracy.
This heterogeneity limits the generalisability of the results.

More broadly, although we have assessed the risk of publication bias as low (k0 = 3), the full text screening stage of this systematic review was notable for the high number of studies (57) that did not provide raw diagnostic data.
this represents a lost opportunity for systhesis of the hard work of colleoagus past \& present into our current understanding of intraoperative diagnsotics.
For this reason, along with previous authors \cite{stjohnDiagnosticAccuracyIntraoperative2017, irwigGuidelinesMetaanalysesEvaluating1994}, we call for future publications of diagnostic studies in head and neck cancer to provide raw diagnostic data---ideally in the form of a confusion matrix---and clear reporting of how the index and reference tests were performed.
