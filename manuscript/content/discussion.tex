Hutan made a compelling argument that the whole paper, but especially the discussion, should be compelling and exciting for clinicians in general and surgeons in particular if we are to get it published in a strong journal.

Consider including unpublished iknife data? Discuss with ZT.

\subsection{Summary of findings}

This study has performed a systematic reviem and meta-analysis of the diagnostic accuracy of established andemerging intraoperative margin assessment techniques.
The results we provide act as a benchmark for comparison with emerging techniques.
The rising incidence of HNSCC, driven by HPV, and the clear desire from patient preference data to avoid recurrence and morbidity associated with adjuvant treatment means that there is a pressing need to develop IMA techniques with satisfactory metrics for widespread adoption into surgical workflow.

Our data suggests that the two best IMA techniques are frozen section (insert sensitivity and specificity here), and chemiluminescent techniques (insert pooled sensitivity and specificity here), with an AUC of ($x$ \& $y$) respectively.
Despite these promising results, neither technique is widely used in current clinical practice
\emph{summary of the reasons for each here before going into detail.}

\subsection{Frozen section}

Frozen section is a long-established adjunct for intra-operative diagnosis and margin evaluation.

In our meta-analysis, we demonstrated excellent specificity for frozen section (check final figures)
However, the sensitivity was less impressive overall.
Despite these impressive numbers, however, frozen section has not become a routine part of head and neck resection surgery, and it is interesting to explore the reasons why.

Errors: the sources of error as either sampling or interpretation errors, the latter being associated with increased technical errors in sample processing and with less experienced pathologists (layfield)

Costs: frozen section requires specialised equipment, and for an experienced histopathologist to be able to be available for 30 minutes at short notice, with implications on their ability to do other routine pathological work for the health service. 
The economic costs of this are considerable: 
It is associated with significant cost, estimated at \$3123 per patient \cite{dinardoAccuracyUtilityCost2000}. 
(This is over \$5200 in today's money (https://www.usinflationcalculator.com))

In addition to the weaknesses of the technique in terms of diagnostic accuracy in the strict sense evaluated in this systematic review, there is a broader question of whether the use of frozen section, and a negative result provided by a frozen section, is able to add to the rates of local control following resection.

In some head and neck surgical contexts, the operation aims to remove the tumour with large margins - salvage laryngectomy and primary tongue resection are key examples. 
In this situation, the rates of positive margins are usually low, and so teh marginal benefit (no pun intended) of the use of frozen section is unlikely to be justified.





\subsection{Tissue-specific fluorescence}

Tumour-specific targeted fluorescence techniques were shown to have excellent diagnostic accuracy, with overall sensitivity of \SI{92.5}{\percent} and specificity of \SI{82.5}{\percent}. 
The AUROC was 0.924.
This is based on 5 studies, reporting 367 diagnostic events that were included in the random-effects model.

Tissue-specific fluorescence techniques use systemic administration of biocompatible fluorophores---most commonly indocyanine green---that concentrate in tumour tissue, allowing localisation of the tumour  by shining a light of appropriate wavelength on the area to elicit a fluorescent response.
There are two broad approaches: either \emph{always on} fluorophores, that are target to tumour tissue by conjugation with targeting molecules, most commonly humanised antibodies, or \emph{activatable} fluorophores that are injected in the \emph{quenched} state, then upon contact with the tumour environment, undergo a change in order to fluoresce only in that environment.

Three of the papers (citations) dealt with ICG annealed to a HNSCC-specific antibody: either cetuximab or panitumumab.
Pros and cons of this: 80-90\% expression of EGFR
Two papers dealt with different methods of activating the fluorophore in the tumour environment, either a nano-polymer that quenches the ICG at physiological pH and unquenches in the acidic tumour microenvironment, or a precursor fluorophore that is activated in the presence of high concentrations of gamma-GT, an enzyme that is overexpressed in HNSCC (check this, slooter et al).

It is notable that this work is mostly being carried out in the USA and the Netherlands.

Spoke with Nynke van den Berg, who said that one of the drawbacks is that the fluorescence only allows penetration down to 5-7 mm (check the exact amounts). 
This prevents recognition of the cancer deeply, though it does perhaps allow the sample to show where the tumour is close? 
This could perhaps be overcome by the addition of a radioactive moiety, though this a) would add to the expense, b) would only be possible with tumour targeting approaches using EG antibodies, rather than tumour activated approaches, and c) comes with the potential risks associated with systemic administration of radioactive substances.

\subsection{Discussion of other techniques}

Other techniques within our meta-analysis were perhaps less compelling.

Touch imprint cytology only contributed three papers to the systematic review, and the results presented wide confidence intervals. 
It may have a role in resource constrained environments - as highlighted in one of the papers (WHICH) - as minimal complex equipment is required. 
However it is hard to see it having a role as the gold standard of care.

Optical techniques showed some promise, but there was a significant amount of heterogeneity within and between studies. 
This is not surprising, as the group contained studies using a wide range of techniques, including \ldots
\begin{itemize}
\item High resolution microendoscopy
\item Narrow band imaging
\item Elastic scattering spectroscopy
\item Coherent Raman scattering spectroscopy
\item Optomagnetic imaging spectroscopy
\item Optical coherence tomography
\end{itemize}


\subsection{Relevance of findings to clinicians, researchers and policy-makers}

The full text screening stage of this systematic review was notable for the high number (check how many) of studies that did not provide raw diagnostic data, or a celar explanation of how diagnostic metrics were calculated, whether per patient, per sample or per test.
this represents a lost opportunity for systhesis of the hard work of colleoagus past \& present into our current understanding of intraoperative diagnsotics.

For this reason, along with previous authors \cite{stjohnDiagnosticAccuracyIntraoperative2017, irwigGuidelinesMetaanalysesEvaluating1994}, we call for all future publications of diagnostic studies in head and neck cancer to be contingent upon provision of raw diagnostic data---ideally in the form of a confusion matrix---and clear reporting of how the index and reference tests were performed.


\subsection{Limitations}

High levels of heterogeneity.
Papers variably reporting per sample or per patient diagnostic accuracy.

In order to meaningfully compare different intraoperative diagnostic modalities, we have elected to group together related but heterogeneous IMA techniques. 
This heterogeneity limits the generalisability of the results.
The 'optical' category is the most heterogeneous, which corresponds to the novelty of this area of research, and the multiple approaches taken by researchers to exploit the differential optical properties of malignant tissue for rapid diagnostics.


\subsection{The role of IMAs}

A question that remains unanswered is whether the use of IMAs to achieve clear margins results in improved oncological outcomes (LOOK DEEPER INTO THIS AND CITE).
The underlying assumption of IMAs is that association between positive surgical margins and poor oncological outcomes is \emph{causative}.
This reasoning assumes that positive margins imply viable malignant cells left \emph{in situ}, with the ability to continue multiplication, local invasion and ultimately acquisition of the mutations necessary to metastasise to distant sites.

Frozen section has been shown in a robust meta-analysis to reduce positive margin rates by 2--3 percentage points \cite{gorpheSystematicReviewMetaanalysis2019}.

However, it is possible that resection with positive margins---where this was not anticipated by the surgeon---reflects a fundamentally aggressive tumour phenotype that is likely to recur whether or not IMAs facilitate resection with clear margins.
There is some evidence to support this theory.
The revision of positive margins to negative guided by frozen section is a major risk factor for local recurrence \cite{ettlt.PositiveFrozenSection2016}, and where multiple resections are required to achieve clear margins, disease-free and overall-survival is significantly worse \cite{mooreTransoralRoboticSurgery2018}
14 to 22\% of patients may have a positive margin in the final specimen despite a negative intraoperative frozen section result \cite{ordAccuracyFrozenSections1997, due.RefiningUtilityRole2016}, probably representing sampling error.
