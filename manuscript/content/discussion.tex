Rough table of contents for discussion:
\begin{itemize}
\item Summary of findings
\item Frozen section
\item Targeted fluorescence
\item Optical techniques
\item The rest
\item Diagnostic accuracy =/= improved outcomes
\item Significance of these results
\item Limitations
\end{itemize}
 
\subsection{Summary of findings}

This study has performed a systematic review and meta-analysis of the diagnostic accuracy of established and emerging intraoperative margin assessment techniques.
The results we provide act as a benchmark for comparison with emerging techniques.
The rising incidence of \gls{hnscc}, driven by \gls{hpv}, and the clear desire from patient preference data to avoid recurrence and morbidity associated with adjuvant treatment means that there is a pressing need to develop \gls{ima} techniques with satisfactory metrics for widespread adoption into surgical workflow.

Standardised metrics of diagnostic performance are important for fair comparison of diagnostic tests.
Sensitivity and specificity are important metrics for understanding the performance of a diagnostic test.
They are widely used, and useful.
Sensitivitiy measures the rate at which the presence of disease is accurately diagnosed. 
A sensitivity of \SI{100}{\percent} implies that the disease will always correctly identify a patient who has the disease.
Conversely, specificity measures the rate at which the \emph{absence} of disease is correctly diagnosed. 
A test with \SI{100}{\percent} specificity would never identify a healthy patient as having the disease.
\footnote{should I create a sidebox/table showing the formulae for sens/spec/fpr/fnr/accuracy?}

In practical usage, no tests have \SI{100}{\percent} specificity and sensitivity.
Adjusting the diagnostic threshold to improve one frequently worsens the other.
This relationship is demonstrated in the \gls{roc} curve.
A test that performs better overall will have a \gls{roc} curve approaching the top left corner, with an \gls{auc} approaching 1.
A test that performs no better than random will have an \gls{auc} of $0.5$.

It is worth noting that, unlike false positive rate and false negative rate, sensitivity and specificity do not take into account the prevalance of disease. 
In the context of \gls{ima} techniques, the patient is known to have cancer, the question is whether the tissue sampled contains cancer or not. 
The rate at which these contain disease is highly variable, and depends on many complex factors, including the skill of the operating surgeon, the ability of the test to provide rapid results, and whether the disease has an aggressive, infiltrative pattern.
Evaluating sensitivity and specificity allows more standardised comparison.

When evaluating the diagnostic performance of \gls{ima} techniques, the purpose of the techniques should be borne in mind. 
The aim of \gls{ima} is to allow surgeons to remove disease entirely, to reduce risks of recurrence and the need for morbid adjuvant treatment.
Inappropriate removal of healthy tissue is to be avoided, but can be considered a secondary objective for \gls{ima} techniques.
As such, sensitivity is probably the most important metric to consider.

Our data suggests that the two best \gls{ima} techniques are frozen section, with pooled sensitivity \SI{79.5}{\percent} and specificity \SI{99.1}{\percent}, and tumour-targeted fluorescence techniques, with pooled sensitivity \SI{95.7}{\percent} and specificity \SI{82.7}{\percent}.
with an AUC of 0.976 and 0.944 respectively.
Despite these promising results, neither technique is widely used in current clinical practice
\footnote{put in a table outlining all the sens and specs for each modality and refer to it here}

\subsection{Frozen section}

Frozen section is a long-established adjunct for intra-operative diagnosis and margin evaluation.
In our meta-analysis, the specificity of frozen section was (\SI{99.1}{\percent}), the sensitivity was (\SI{79.5}{\percent}).
However, frozen section has not become a routine part of head and neck resection surgery.

A key reason for the limited uptake of frozen section is that the major source of error is not accounted for by sensitivity and specificity metrics.
For any given sample, errors of processing and interpretation are rare.
Instead, most errors occur at the sampling stage.
A common approach is for the surgeon to complete the tumour resection, then if concerns remain about risk of residual disease, they take samples from the tumour bed for frozen section analysis\cite{layfieldFrozenSectionEvaluation2018}.
This risks sampling healthy tissue adjacent to an inconspicuous nidus of residual disease \cite{due.RefiningUtilityRole2016}, or miscommunication between surgical and pathological teams \cite{blackc.CriticalEvaluationFrozen2006}.
These errors do not affect the sensitivity and specificity of frozen section, but markedly reduce its utility as an \gls{ima}.
Sampling errors can be reduced by taking a specimen-driven approach \cite{maxwellEarlyOralTongue2015}, where the whole surface of the specimen is sampled.
However, this substantially increases the time and cost per analysis.

Frozen section requires specialised equipment, and for an experienced histopathologist to be able to be available for \SIrange{20}{30}{\minute} at short notice, with implications on their ability to do other routine pathological work for the health service. 
The economic costs of this are considerable: 
It is associated with significant cost, estimated at \$3123 per patient \cite{dinardoAccuracyUtilityCost2000}. 
(This is over \$5200 in today's money (https://www.usinflationcalculator.com))

\footnote{Note that \emph{all} of the studies that were retrospective were frozen section studies, which raises questions about the conclusions and comparisons that can be drawn from these data.}

\subsection{Tissue-specific fluorescence}

Tissue specific fluorencence is an exciting emerging technique that uses systemic administration of biocompatible fluorophores---most commonly indocyanine green---that preferentially accumulate in tumour tissue.
This allows tumour visualisation of the tumour by shining a light of appropriate wavelength on the area to elicit a fluorescent response.
In our meta-analysis, tumour-specific targeted fluorescence techniques had excellent diagnostic accuracy, with overall sensitivity of \SI{95.7}{\percent} and specificity of \SI{82.7}{\percent}. 
The \gls{auc} was 0.944.
As is to be anticipated with an emerging technique, all of the included studies were pilot or phase I studies, with limited patient recruitment. 
However, the methodology and reporting was of a high standard, with the work being carried out mostly in reputable units in the USA and the Netherlands.

In addition to the excellent sensitivity, this approach had resolution able in some cases to identify otherwise undetectable satellite metastases \cite{voskuilf.j.ImageguidedSurgeryTumor2019}, and allows detection of fluorescence signal through up to \SI{6}{\milli\metre} of tissue, potentially allowing estimation of the proximity of the tumour to the resection margin, albeit with reduced specificity \cite{vankeulenRapidNoninvasiveFluorescence2019}.
These results suggest that targeted fluorescence techniques may have a major role in the future of head \& neck surgical oncology.

Included in the meta-analysis were two different techniques for targeting the fluorescence within the tumour.
The majority linked the fluorophore to monoclonal antibodies  targeted at \gls{egfr}, which is almost always overexpressed in \gls{hnscc}\cite{chungIncreasedEpidermalGrowth2016}.
The fluorescent antibody binds to the surface of cells expressing \gls{egfr}, causing substantially increased fluorescent signal in tumour compared with adjacent normal tissue.

The other approach was conditional \textit{quenching}, whereby the fluorophore is enveloped in a nanoparticle micelle that inhibits fluorescence\cite{voskuilExploitingMetabolicAcidosis2020}.
This micelle is stable until it encounters a pH below a tunable threshold, at which point the micelle irreversibly dissociates. 
When administered intravenously, the micelle remains stable until it reaches the hypoxic, low-pH tumour microenvironment. 
Here, the micelle disociates, releasing the fluorophore, resulting in tissue-specific fluorescence.

%%%advantages and disadvantages of each technique and the overall approach
Each of these two approaches have their advantages and disadvantages.
The monoclonal antibodies used in these studies (cetuximab \cite{warramFluorescenceImagingLocalize2016, voskuilf.j.FluorescenceguidedImagingResection2020}and panitumumab \cite{gaor.w.DeterminationTumorMargins2018, vankeulenRapidNoninvasiveFluorescence2019}) are already in clinical use, with a well established safety profile.
Repurposing already existing drugs provides greater confidence in the safety and side-effects profile, as well as speeding up development. 

Another advantage of using monoclonal antibodies, is that they could be combined with other labelling modalities.
One possibility is the use of radio-labelling, allowing localisation of tumour deposits at depths greater than the 6 mm allowed by fluorescence, comparable to the process of sentinel node biopsy.
However, this approach could not be used with nanoparticle micelles, as the quenching would have no effect on the transmission of gamma radiation, so signal would not be concentrated within the tumour.

%how about the effect of hypoxia?
%antibodies bad, nanoprobes, essential!

The use of monoclonal antibodies targeting \gls{egfr} are limited bythe requirement for the tumour to express \gls{egfr}.
This is usually significantly upregulated in \gls{hnscc}, but can be suppressed in some tumour areas, particularly where the tumour microenvironment is hypoxic \cite{mayerDownregulationEGFRHypoxic2016}.
the hypoxic microenvironment is near ubiquitous, making this a concerning feature.
The ubiquity of hyppoxia makes the alternative approach appealing.

This approach exploits the localised acidosis that is a near ubiquitous hallmark of solid cancers, and thus has the advantage of being 'tumour agnostic'.

One weakness of tissue-targeted fluorescence as it is used in these studies is that it is difficult to use in situ as the operation is progressing. 
The investigators carefully examined the possibility of using the fluorescence to guide resection \cite{mooreCharacterizingUtilityLimitations2017}, however there was 87\% lower signal in vivo compared with imaging of the resected specimen, and repeated interruption to surgical workflow with the need to dim operative lights and pause to take fluorescence imaging made this approach inaccurate and impractical. 
The authors now routinely take the resection specimen to an analysis box on a back table in theatre for analysis \cite{gaor.w.DeterminationTumorMargins2018}. 
This loses some of the immediacy of real-time feedback, but remains much faster than frozen section.
Akin to specimen driven frozen section, so fine in that regard, but ultimately it is indirect information about the likelihood of viable tumour cells being left in situ.

\subsection{Touch Imprint Cytology}

Other techniques within our meta-analysis were perhaps less compelling.

Touch imprint cytology only contributed three papers to the systematic review, and the results presented wide confidence intervals. 
It may have a role in resource constrained environments - as highlighted in one of the papers (WHICH) - as minimal complex equipment is required. 
However it is hard to see it having a role as the gold standard of care.

\subsection{Tissue staining techniques}

Application of dyes that differentially stain malignant and normal tissue have been known about for some time, but still are not in widespread use.
Lugol's iodine has been proposed as a useful adjunct for identifying dysplastic tissue from normal mucosa, by iodine staining starches present in normal tissue and absent in dysplastic tissue.
It has been investigated in a clinical trial, but the results have not yet been published (cite abstracts and clinical trial number).
Like all other stains, it is only of use at the start of the procedure, as it interacts unpredictably with the presence of blood in the operating field, making it impossible to accurately tumour from normal tissue at the deep margin consisting of dissected muscle and connective tissue.

Toluidine blue is a tissue dye - often used in examination of fixed tissue slides - that stains nucleic acids blue.
The high abundance of nucleic acids in rapidly multiplying malignant tissue means that it stains more intensely than adjacent normal tissue.
TODO---how is it at distinguishing malignancy and dysplasia? this should be a clear criticism. 
Also doesn't do teh deep margin

5-ALA (Leunig) and the gamma-gt spray (slooter) were both interesting - these stains were both stains that contained a fluorescent dye that was activated only in the presence of tumour. 
, or a precursor fluorophore that is activated in the presence of high concentrations of gamma-GT, an enzyme that is overexpressed in HNSCC (check this, slooter et al).

\subsection{Optical techniques}

Optical techniques as grouped here all measure the signal returned when light is shone on the target tissue.
Optical techniques showed some promise, but there was a significant amount of heterogeneity within and between studies. 
This is not surprising, as the group contained studies using a wide range of techniques, including (internet definitions) \ldots

High resolution microendoscopy \cite{vilap.m.DiscriminationBenignNeoplastic2012, milesb.a.OperativeMarginControl2015}\footnote{define HRME}
Narrow band imaging \cite{tirelliNarrowBandImaging2015, tirellig.TailoredResectionsOral2018}Narrowband imaging uses two specific wavelengths in the visible light spectrum to highlight the interpapillary capillary loop patterns in the mucosa, which are notably disorganised in tumour tissue \cite{vuEfficacyNarrowBand2014}.
Elastic scattering spectroscopy \cite{grilloneg.a.ColorCancerMargin2017}\footnote{define elastic scattering spectroscopy}
Coherent Raman scattering spectroscopy \cite{hoeslir.c.CoherentRamanScattering2017} Raman spectroscopy quantifies the inelastic scattering of photons from tissue illuminated with monochromatic light.
Optomagnetic imaging spectroscopy \cite{lisulb.PredictiveValueOptomagnetic2019}\footnote{define optomagnetic imaging spectroscopy} 
Multimodal nonlinear microscopy \cite{heukes.MultimodalNonlinearMicroscopy2016}\footnote{define multimodal nonlinear microscopy---can this be bundled together with HRME?} 
Optical coherence tomography (wikipedia) Optical coherence tomography (OCT) is an imaging technique that uses low-coherence light to capture micrometer-resolution, two- and three-dimensional images from within optical scattering media (e.g., biological tissue) \cite{hamdoonz.OpticalCoherenceTomography2016} \Gls{oct} uses changes in tissue optical properties---such as refractive index---to generate real-time, high-resolution cross-sectional images of tissue \cite{heidaria.e.UseOpticalCoherence2020}.
Tissue autofluorescence \cite{ohnishiy.UsefulnessFluorescenceVisualization2016} Autofluorescence measures the photons emitted by fluorescence when monochromatic light is shone on mucosal tissue; a property which is lost by abnormal mucosa \cite{leey.-j.IntraoperativeFluorescenceGuidedSurgery2020}\footnote{check this citation}.

Maybe put in a dig here about narrow band imaging and the Velscope. 
Particularly gun for peter thompson.

\subsection{Relevance of findings to clinicians, researchers and policy-makers}

Hutan made a compelling argument that the whole paper, but especially the discussion, should be compelling and exciting for clinicians in general and surgeons in particular if we are to get it published in a strong journal.

The full text screening stage of this systematic review was notable for the high number (check how many) of studies that did not provide raw diagnostic data, or a celar explanation of how diagnostic metrics were calculated, whether per patient, per sample or per test.
this represents a lost opportunity for systhesis of the hard work of colleoagus past \& present into our current understanding of intraoperative diagnsotics.

For this reason, along with previous authors \cite{stjohnDiagnosticAccuracyIntraoperative2017, irwigGuidelinesMetaanalysesEvaluating1994}, we call for all future publications of diagnostic studies in head and neck cancer to be contingent upon provision of raw diagnostic data---ideally in the form of a confusion matrix---and clear reporting of how the index and reference tests were performed.

Call for a clinical trial of targeted fluorescence?

\subsection{Limitations}

High levels of heterogeneity.
Papers variably reporting per sample or per patient diagnostic accuracy.

In order to meaningfully compare different intraoperative diagnostic modalities, we have elected to group together related but heterogeneous IMA techniques. 
This heterogeneity limits the generalisability of the results.
The 'optical' category is the most heterogeneous, which corresponds to the novelty of this area of research, and the multiple approaches taken by researchers to exploit the differential optical properties of malignant tissue for rapid diagnostics.


\subsection{The role of IMAs}

Improved diagnostic accuracy does not necessarily result in improved rates of negative margins, or indeed improved survival outcomes.
The limitations of frozen section illustrate this point clearly.
In UK practice, the NICE\footnote{acronymify} guidelines suggest frozen section be used where there is clinical suspicion of positive margins (cite).
In this context, the surgeon would usually take a sample from the wound bed.
The high specificity and sensitivity of frozen section means that it is likely to provide a correct diagnosis \textit{for that sample} unless there are technical failures during sample processing (CITE).
However, the success of this approach relies on the surgeon correctly identifying and sampling residual disease in the resection defect. 
If the surgeon inadvertently sample normal tissue adjacent to residual disease, the patient would have positive margins (with the attendent morbidity and mortality) despite frozen section correctly reporting a negative result.
This \textit{sampling error} is a key weakness of defect-drive frozen section (CITE), and is a key limitation to its wider adoption.

With this in mind, any new \gls{ima} technique should be able to provide results for the entiretyl of the specimen and/or the resection defect with high resolution whilst retaining excellent sensitivity and specificity.

This would be a good point to sing the praises of targeted fluorescence.
Also could float the idea of combining it, or some other targeted approach as a label for the iKnife.


Whatever the case is, ideally any and all new \gls{ima} techniques should be evaluated on their ability to reduce definitive positive margin rates, and to reduce rates of recurrence and disease specific morbidity.

Frozen section has been shown in a robust meta-analysis to reduce positive margin rates by 2--3 percentage points \cite{gorpheSystematicReviewMetaanalysis2019}.

%%%%%%%%%%%%%%%
In addition to the weaknesses of the technique in terms of diagnostic accuracy in the strict sense evaluated in this systematic review, there is a broader question of whether the use of frozen section, and a negative result provided by a frozen section, is able to add to the rates of local control following resection.

There is some evidence that the use of frozen section in transoral surgery reduces the rate of definitive positive margins (systematic review)\cite{gorpheSystematicReviewMetaanalysis2019}. This is also covered in the summary section below.
%%%%%%%%%%%%%%%%1

However, it is possible that resection with positive margins---where this was not anticipated by the surgeon---reflects a fundamentally aggressive tumour phenotype that is likely to recur whether or not IMAs facilitate resection with clear margins.
There is some evidence to support this theory.
The revision of positive margins to negative guided by frozen section is a major risk factor for local recurrence \cite{ettlt.PositiveFrozenSection2016}, and where multiple resections are required to achieve clear margins, disease-free and overall-survival is significantly worse \cite{mooreTransoralRoboticSurgery2018}
14 to 22\% of patients may have a positive margin in the final specimen despite a negative intraoperative frozen section result \cite{ordAccuracyFrozenSections1997, due.RefiningUtilityRole2016}, probably representing sampling error.

Consider including unpublished iknife data? Perhaps the daoms abstract from Jag? Discuss with ZT.

