%Provide the rationale for the review in the context of what is already known

\Gls{hnscc} is the 7th most common form of cancer worldwide. 
It is associated with alcohol and tobacco consumption, as well as high-risk forms of \gls{hpv}.
it has a high mortality rate, and significant morbidity, including pain, disfigurment, speech and swallowing dysfunction and psychosocial distress. \cite{chowHeadNeckCancer2020}.

Treatment of head and neck cancer varies according to disease characteristics, notably the subsite of origin.
Options include primary \gls{crt}, or surgery followed by adjuvant \gls{crt} in high risk cases.
Survival outcomes are comparable, so functional outcomes dominate treatment selection.
Oral squamous cell carcinoma, locally recurrent \gls{hnscc} and --- increasingly --- early oropharyngeal tumours are usually offered surgery, as the morbidity is lower than that of \gls{crt}, especially in patients who have had previous \gls{crt}\footnote{[CITATIONS NEEDED]}.

The success of surgical resection relies on the complete removal of the tumour.
The current gold standard for assessing the completeness of resection is histopathological assessment of the \gls{ffpe} resection specimen.\footnote{[CITATION NEEDED]}
If malignant cells are present on the cut surface of the specimen, the implication is that the tumour was transected during resection, and thus that viable tumour cells are left \textit{in situ}.
This is referred to as \emph{positive margins}.

Estimates of the incidence of positive margins vary.
Reports suggest positive margin rates overall of $x$\%\footnote{find good data to cite here}, whilst the British Association of Head and Neck Oncologists\footnote{acronymify? Also, american guidelines?} suggest that best practice would have a positive margin rate no higher than 5\%\footnote{check and cite}.

Resection with positive margins is an extremely poor prognostic indicator.
The increased risk of recurrence means that eligible patients are offered adjuvant chemoradiotherapy, adding the morbidity of these treatments to that of surgery.
radiotherapy to the head and neck frequently leads to poor speech and swallow outcomes \cite{machtayFactorsAssociatedSevere2008, wangPharyngoesophagealStrictureTreatment2012}, which is one of the most important outcomes for patients following head and neck cancer treatment \cite{wilsonDysphagiaNonsurgicalHead2011} 
Despite this additional treatment, patients still have a higher risk of recurrence and death from disease.

\paragraph{Why do positive margins occur?}
Limitations of surgical ability to identify tumour from normal, esp at the deep margin

%Current practice is for surgeons to identify the extent of the primary tumour by its macroscopic appearance

The current standard of practice is patients to be scheduled for surgery following a pre-operative tissue biopsy confirming the diagnosis of cancer, and pre-operative imaging both staging the tumour and giving an indication of its anatomical location and extent.
During the operation, the surgeon relies on the visual and tactile properties of the tumour (erythema, swelling, ulceration, feeling harder than surrounding normal mucosa) to determine the extent of the lesion and plan the resection margin to allow an adequate cuff of normal tissue on the specimen.

There are significant limitations to this approach. 
Tumour extent may not always be macroscopically visible, and the tactile and kinaesthetic proprieties of tumours vary substantially (CITATION?).

This is borne out by the evidence: even under highly controlled conditions, surgical experience and acumen appears to have a sensitivity of 88.9\% and specificity of 81.1\% \footnote{change this to the siunitx format} \cite{chaturvedip.GrossExaminationSurgeon2014} for the \emph{mucosal} margin.
Surgical performance at the deep margin is even poorer, with positive margins occuring at this margin alone in 17\% of cases of oral cancer \cite{woolgarHistopathologicalAppraisalSurgical2005}.

\subsection{What intraoperative margin assessment tools exist?}

Intraoperative margin assessment tools may improve oncological outcomes.
rationale for IMA

Introductory paragraph explaining IMA tools and introducing the overarching categories so that I can introduce them individually in the paragraphs below

\paragraph{Frozen section}

Frozen section is a very long-standing technique.
It involves rapid fixing, staining and histological assessment of tissue.
There is considerable flexibility in how this is used.
After resection of the main specimen, the surgeon can take samples from the tumour bed, targiteing areas that they feel are at risk.
This is known as a \emph{``defect-driven''} approach.
Alternatively, the entire resection specimen can be sent to the pathologist, who can then remove the ``faces'' of te specimen and examine them for tumour: the \emph{``specimen-driven''} approach.

\paragraph{Touch Imprint Cytology}
A simple but effective technique. 
Touch imprint cytology (TIC) is a simple technique in which the cut face of the specimen is pressed to a glass slide. 
Cells from the cut surface adhere, then are fixed, stained and microscopically examined. 
This offers an indication of whether malignant cells are present at the specimen margin \cite{naveedDiagnosticAccuracyTouch2017}.


\paragraph{Optical techniques}
A number of techniques exploit the differing optical properties of cancer to allow intraoperative margin assessment.
Narrowband imaging uses filters to restrict to only two specific wavelengths in the visible light spectrum.
Using this to illuminate the tissues highlights the interpapillary capillary loop patterns in the mucosa, which are notably disorganised in tumour tissue \cite{vuEfficacyNarrowBand2014}.
Optical techniques as grouped here all measure the signal returned when light is shone on the target tissue.
Autofluorescence with the Velscope measures the photons emitted by fluorescence when monochromatic light is shone on the tissue.
Optical coherence tomography measures the refractive properties of tissue to construct a composite image of the tissue in 3D.
Raman spectroscopy quantifies the inelastic scattering of photons from tissue illuminated with monochromatic light.

Optical coherence tomography (OCT) uses changes in tissue optical properties---such as refractive index---to generate real-time, high-resolution cross-sectional images of tissue \cite{heidaria.e.UseOpticalCoherence2020}.

Autofluorescence techniques exploit the property of normal tissue to autofluoresce. 
It absorbs light and then emits photons at a longer wavelength \cite{leey.-j.IntraoperativeFluorescenceGuidedSurgery2020}.
Abnormal mucosa loses this property, and can thus be distinguished in real time.

\paragraph{Chemiluminescence}

A significant advance in recent years is the development of targeted chemiluminescent techniques, in which a tumour-specific marker is linked to a luminescent dye. 
This `smart' dye is then injected into the patient, and preferentially accumulates within the tumour. 

\subsection{Cutting edge IMA tools}

Here talk about future techniques, such as the iknife. 
This may end up being better in the discusssion.

A number of techniques are emerging, and it is important to establish the current benchmark for diagnostic performance against which innovative approaches may be evaluated.

This study aims to quantify the accuracy of tools and techniques used during radical surgical resection of head and neck squamous cell carcinoma to identify tissue types with the aim of identifying and correcting close or positive margins.

In the absence of sufficient studies with oncological outcomes to identify superior IMA techniques, we have gone for diagnostic outcome measures. Move this to discussion?
