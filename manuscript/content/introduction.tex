
% Rationale
Provide the rationale for the review in the context of what is already known

Head and neck cancer is common

Surgical resection of head and neck cancer is the treatment of choice in some contexts
Particularly primary oral cancer and local recurrence.

For x percent of patients, they will have positive margins. 
This is higher in some contexts than others.
Resection with positive margins is an extremely poor prognostic indicator

These patients will go on to have adjuvant radiotherapy or chemoradiotherapy.
If they have already had these, they can't have more.
If they do have adjuvant treatment, they still have higher risk of recurrence and death from disease, and also have the morbidity of multi-modality therapy.
Downsides of adjuvant treatment in more detail here.

Why are margins positive?
Limitations of surgical ability to identify tumour from normal, esp at the deep margin



Intraoperative margin assessment tools may improve oncological outcomes.
rationale for IMA

\subsection{Current IMA tools}

Introductory paragraph explaining IMA tools and introducing the overarching categories so that I can introduce them individually in the paragraphs below

\paragraph{Frozen Section}



...

\subsection{Cutting edge IMA tools}

Here talk about future techniques, such as the iknife. 
This may end up being better in the discusssion.
...



% Objectives
Provide an explicit statement of questions being addressed with reference to participants, interventions, comparisons, outcomes, and study design (PICOS). 

This study aims to quantify the accuracy of tools and techniques used during radical surgical resection of head and neck squamous cell carcinoma to identify tissue types with the aim of identifying and correcting close or positive margins.


In the absence of sufficient studies with oncological outcomes to identify superior IMA techniques, we have gone for diagnostic outcome measures.
