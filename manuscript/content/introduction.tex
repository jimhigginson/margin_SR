%Provide the rationale for the review in the context of what is already known

\Gls{hnscc} is the 7th most common form of cancer worldwide. 
It is associated with alcohol and tobacco consumption, as well as high-risk forms of \gls{hpv}.
it has a high mortality rate, and significant morbidity, including pain, disfigurment, speech and swallowing dysfunction and psychosocial distress. \cite{chowHeadNeckCancer2020}.

Current standard treatment of \gls{hnscc} is usually either primary \gls{crt}, or surgical resection. 
High risk surgical cases are usually offered post-operative \gls{crt}.
Both of these treatement approaches have comparable survival, so post-treatment morbidity is a key influence on treatment decision-making.
Small primary \gls{hnscc} tumours, operable oral cancer, and locally recurrent \gls{hnscc} are usually offered surgery\cite{niceCancerUpperAerodigestive2018, kerawalaOralCavityLip2016}.

The success of surgical resection relies on the complete removal of the tumour.
The current gold standard for assessing the completeness of resection is histopathological assessment of the \gls{ffpe} resection specimen\cite{helliwellPathologicalAspectsAssessment2016}.
Malignant cells present on the cut surface of this specimen imply that tumour was transected during resection, and therefore that viable tumour cells are left \textit{in situ}.
This is referred to as \emph{positive margins}.

Resection with positive margins is an extremely poor prognostic indicator\cite{hinniSurgicalMarginsHead2013a}.
The increased risk of recurrence means that eligible patients are offered adjuvant chemoradiotherapy, adding the morbidity of these treatments to that of surgery.
Adjuvant radiotherapy to the head and neck frequently leads to poor speech and swallow outcomes \cite{machtayFactorsAssociatedSevere2008, wangPharyngoesophagealStrictureTreatment2012}, which patients report as priority outcomes following head and neck cancer treatment\cite{wilsonDysphagiaNonsurgicalHead2011}.
Despite this additional treatment, patients still have a higher risk of recurrence and death from disease.

The reasons for positive margins are complex and multifactorial \cite{hinniSurgicalMarginsHead2013a, upileUncertaintySurgicalMargin2007}, however it remains the one prognostic factor in \gls{hnscc} that is under the direct control of the treating team.
During the operation, surgeons rely on the macroscopic visual and tactile properties of the tumour to determine the extent of the lesion and plan the resection margin.
There are significant limitations to this approach. 
Even under highly controlled conditions, surgical experience and acumen appears to have a sensitivity of \SI{88.9}{\percent} and specificity of \SI{81.1}{\percent} for the \emph{mucosal} margin\cite{chaturvedip.GrossExaminationSurgeon2014}.
Surgical performance at the deep margin is even poorer, with positive margins occuring at this margin alone in \SI{17}{\percent} of cases of oral cancer \cite{woolgarHistopathologicalAppraisalSurgical2005}.

An \gls{ima} technique that can address these shortcomings and improve positive margin rates is a major goal of surgical oncology research.
Such a technique would, ideally, provide an accurate diagnosis of tissue the surgeon suspects of being malignant, and allow a complete resection with clear margins.
Results should be provided rapidly to avoid delays to operative time.
A wide variety of \gls{ima} techniques have been proposed, however, no single technique meets these criteria completely, and as such none have met with widespread acceptance by head and neck surgeons.

Frozen section is the longest-standing \gls{ima} technique, and is widely used, though mostly for selected cases where margins are of particular concern.
After resection of the main specimen, the surgeon can take samples from the tumour bed, targeting  areas that they feel are at risk.
This is known as a \emph{``defect-driven''} approach.
Alternatively, the entire resection specimen can be sent to the pathologist, who can then remove the ``faces'' of the specimen and examine them for tumour: the \emph{``specimen-driven''} approach.

A number of \gls{ima} techniques employ stains or dyes that exploit structural or metabolic differences between normal and pathological mucosa to differentially colour abnormal mucosa, helping to guide resection\cite{mccaulj.a.LIHNCSLugolIodine2013, allegraEarlyGlotticCancer2020}

Touch imprint cytology (TIC) is a simple technique in which the cut face of the specimen is pressed to a glass slide. 
Cells from the cut surface adhere, then are fixed, stained and microscopically examined. 
This offers an indication of whether malignant cells are present at the specimen margin \cite{naveedDiagnosticAccuracyTouch2017}.

Targeted fluorescence techniques couple an injectable fluorescent dye with a tumours-specific targeting mechanisms.
Tumour tissue then displays a fluorescent signal that is readily distinguised from background by dedicated detectors.
This signal can be used to guide resection, evaluate the tumour bed for residual disease after resection, or rapidly assess the margins of the resected specimen.

A number of related techniques exploit the differing optical properties of cancer to allow intraoperative margin assessment.
These optical techniques employ light of a variety of wavelengths, and detect the differential emission of electromagnetic radiation from tumour and normal tissue. 
The detection element is variable, and can be as simple as visual inspection of tissue fluorescence, or as complex as computational discrimination of inelastic photon scattering.
Some techniques, notably VelScope autofluorescence and narrow-band imaging are in clinical use.
Other optical techniques remain in their infancy.

In addition to these more established techniques, a number of exciting possibilities are emerging. 
Intraoperative mass spectrometry, exploiting the different metabolic features of malignant tissue, has been demonstrated using a number of different acquisition techniques in solid tumours including breast, colorectal and brain tumours\cite{tzafetasIntelligentKnifeIKnife2020, masonLipidomicProfilingColorectal2021, balogIntraoperativeTissueIdentification2013}.

It is important to establish the current benchmark for diagnostic performance against which innovative approaches may be evaluated.
This study aims to quantify the accuracy of tools and techniques used during radical surgical resection of head and neck squamous cell carcinoma to identify tissue types with the aim of identifying and correcting close or positive margins.

