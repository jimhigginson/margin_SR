\section{Introduction}


% Rationale
Provide the rationale for the review in the context of what is already known

Head and neck cancer is common

Surgical resection of head and neck cancer is the treatment of choice in some contexts

Resection with positive margins is an extremely poor prognostic indicator

Intraoperative margin assessment tools may improve oncological outcomes.



% Objectives
Provide an explicit statement of questions being addressed with reference to participants, interventions, comparisons, outcomes, and study design (PICOS). 

This study aims to quantify the accuracy of tools and techniques used during radical surgical resection of head and neck squamous cell carcinoma to identify tissue types with the aim of identifying and correcting close or positive margins.


In the absence of sufficient studies with oncological outcomes to identify superior IMA techniques, we have gone for diagnostic outcome measures.
