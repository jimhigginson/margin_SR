%Provide the rationale for the review in the context of what is already known

\Gls{hnscc} is the 7th most common form of cancer worldwide. 
It is associated with alcohol and tobacco consumption, as well as high-risk forms of \gls{hpv}.
it has a high mortality rate, and significant morbidity, including pain, disfigurment, speech and swallowing dysfunction and psychosocial distress. \cite{chowHeadNeckCancer2020}.

Treatment of head and neck cancer varies according to disease characteristics, notably the subsite of origin.
Options include primary \gls{crt}, or surgery followed by adjuvant \gls{crt} in high risk cases.
Survival outcomes are comparable, so functional outcomes dominate treatment selection.
Oral squamous cell carcinoma, locally recurrent \gls{hnscc} and --- increasingly --- early oropharyngeal tumours are usually offered surgery, as the morbidity is lower than that of \gls{crt}, especially in patients who have had previous \gls{crt}\footnote{[CITATIONS NEEDED]}.

The success of surgical resection relies on the complete removal of the tumour.
The current gold standard for assessing the completeness of resection is histopathological assessment of the \gls{ffpe} resection specimen.\footnote{[CITATION NEEDED]}
If malignant cells are present on the cut surface of the specimen, the implication is that the tumour was transected during resection, and thus that viable tumour cells are left \textit{in situ}.
This is referred to as \emph{positive margins}.
Estimates of the incidence of positive margins vary.
Reports suggest positive margin rates overall of $x$\%\footnote{find good data to cite here}, whilst the British Association of Head and Neck Oncologists\footnote{acronymify? Also, american guidelines?} suggest that best practice would have a positive margin rate no higher than 5\%\footnote{check and cite}.

Resection with positive margins is an extremely poor prognostic indicator.
The increased risk of recurrence means that eligible patients are offered adjuvant chemoradiotherapy, adding the morbidity of these treatments to that of surgery.
radiotherapy to the head and neck frequently leads to poor speech and swallow outcomes \cite{machtayFactorsAssociatedSevere2008, wangPharyngoesophagealStrictureTreatment2012}, which is one of the most important outcomes for patients following head and neck cancer treatment \cite{wilsonDysphagiaNonsurgicalHead2011} 
Despite this additional treatment, patients still have a higher risk of recurrence and death from disease.

The reasons for positive margins are complex and multifactorial, however it remains the one prognostic factor in \gls{hnscc} that is under the direct control of the treating team.
During the operation, the surgeon relies on the macroscopic visual and tactile properties of the tumour to determine the extent of the lesion and plan the resection margin.
There are significant limitations to this approach. 
Tumour extent may not always be macroscopically visible, and the tactile and kinaesthetic proprieties of tumours vary substantially (CITATION?).
Even under highly controlled conditions, surgical experience and acumen appears to have a sensitivity of 88.9\% and specificity of 81.1\% \footnote{change this to the siunitx format} \cite{chaturvedip.GrossExaminationSurgeon2014} for the \emph{mucosal} margin.
Surgical performance at the deep margin is even poorer, with positive margins occuring at this margin alone in 17\% of cases of oral cancer \cite{woolgarHistopathologicalAppraisalSurgical2005}.

To address these shortcomings, a long-standing goal of surgical oncology research is the search for an \gls{ima}  technique.
Such a technique would, ideally, provide an accurate diagnosis of tissue the surgeon suspects of being malignant, and allow a complete resection with clear margins.
This would ideally be completed rapidly to avoid an increase in operative time.
A wide variety of \gls{ima} techniques have been proposed, with some finding proponents in some units.
However, no single \gls{ima} technique meets these criteria completely, and as such none have met with widespread acceptance by head and neck surgeons.

Frozen section is the longest-standing \gls{ima} technique, and is widely used, though mostly for selected cases where margins are of particular concern.
After resection of the main specimen, the surgeon can take samples from the tumour bed, targeting  areas that they feel are at risk.
This is known as a \emph{``defect-driven''} approach.
Alternatively, the entire resection specimen can be sent to the pathologist, who can then remove the ``faces'' of te specimen and examine them for tumour: the \emph{``specimen-driven''} approach.

Touch imprint cytology (TIC) is a simple technique in which the cut face of the specimen is pressed to a glass slide. 
Cells from the cut surface adhere, then are fixed, stained and microscopically examined. 
This offers an indication of whether malignant cells are present at the specimen margin \cite{naveedDiagnosticAccuracyTouch2017}.

\paragraph{Optical techniques}
A number of related techniques exploit the differing optical properties of cancer to allow intraoperative margin assessment.
Optical techniques as grouped here all measure the signal returned when light is shone on the target tissue.
Narrowband imaging uses two specific wavelengths in the visible light spectrum to highlight the interpapillary capillary loop patterns in the mucosa, which are notably disorganised in tumour tissue \cite{vuEfficacyNarrowBand2014}.
Autofluorescence measures the photons emitted by fluorescence when monochromatic light is shone on mucosal tissue; a property which is lost by abnormal mucosa \cite{leey.-j.IntraoperativeFluorescenceGuidedSurgery2020}\footnote{check this citation}.
\Gls{oct} uses changes in tissue optical properties---such as refractive index---to generate real-time, high-resolution cross-sectional images of tissue \cite{heidaria.e.UseOpticalCoherence2020}.

Raman spectroscopy quantifies the inelastic scattering of photons from tissue illuminated with monochromatic light.

\begin{itemize}
\item High resolution microendoscopy
\item Elastic scattering spectroscopy
\item Coherent Raman scattering spectroscopy 
\item Optomagnetic imaging spectroscopy
\end{itemize}

\paragraph{Topical staining techniques}
A number of \gls{ima} techniques employ stains or dyes that differentially colour abnormal mucosa.
toluidine, lugols.

\paragraph{Targeted fluorescence}
A significant advance in recent years is the development of targeted fluorescence techniques, in which a tumour-specific marker is linked to a fluorescent agent. 
This `smart' dye is then injected into the patient, and preferentially accumulates within the tumour. 

\subsection{Cutting edge IMA tools}
In addition to these more established techniques, a number of exciting possibilities are emerging. 
Intraoperative mass spectrometry, exploiting the different metabolic features of malignant tissue, has been demonstrated using a number of different acquisition techniques in solid tumours including breast, colorectal and brain tumours.

A number of techniques are emerging, and it is important to establish the current benchmark for diagnostic performance against which innovative approaches may be evaluated.

This study aims to quantify the accuracy of tools and techniques used during radical surgical resection of head and neck squamous cell carcinoma to identify tissue types with the aim of identifying and correcting close or positive margins.

In the absence of sufficient studies with oncological outcomes to identify superior IMA techniques, we have gone for diagnostic outcome measures. Move this to discussion?
