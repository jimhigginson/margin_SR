
% Rationale
Provide the rationale for the review in the context of what is already known

\Gls{hnscc} is the 7th most common form of cancer worldwide. 
It is associated with alcohol and tobacco consumption, as well as high-risk forms of \gls{hpv}.
It has a high mortality rate, and significant morbidity, including pain, disfigurment, speech and swallowing dysfunction and psychosocial distress. \cite{chowHeadNeckCancer2020}.



Surgical resection of head and neck cancer is the treatment of choice in some contexts
Particularly primary oral cancer and local recurrence.

For x percent of patients, they will have positive margins. 
This is higher in some contexts than others.
Resection with positive margins is an extremely poor prognostic indicator

These patients will go on to have adjuvant radiotherapy or chemoradiotherapy.
If they have already had these, they can't have more.
If they do have adjuvant treatment, they still have higher risk of recurrence and death from disease, and also have the morbidity of multi-modality therapy.
Downsides of adjuvant treatment in more detail here.

\paragraph{Why do positive margins occur?}
Limitations of surgical ability to identify tumour from normal, esp at the deep margin


\subsection{What intraoperative margin assessment tools exist?}

Intraoperative margin assessment tools may improve oncological outcomes.
rationale for IMA

Introductory paragraph explaining IMA tools and introducing the overarching categories so that I can introduce them individually in the paragraphs below

\paragraph{Frozen section}

Frozen section is a very long-standing technique.
It involves rapid fixing, staining and histological assessment of tissue.
There is considerable flexibility in how this is used.
After resection of the main specimen, the surgeon can take samples from the tumour bed, targiteing areas that they feel are at risk.
This is known as a \emph{``defect-driven''} approach.
Alternatively, the entire resection specimen can be sent to the pathologist, who can then remove the ``faces'' of te specimen and examine them for tumour: the \emph{``specimen-driven''} approach.

\paragraph{Touch Imprint Cytology}
A simple but effective technique. 
Touch imprint cytology (TIC) is a simple technique in which the cut face of the specimen is pressed to a glass slide. Cells from the cut surface adhere, then are fixed staned and microscopically examined. This offers an indication of whether malignant cells are present at the specimen margin \cite{naveedDiagnosticAccuracyTouch2017}.


\paragraph{Optical techniques}
A number of techniques exploit the differing optical properties of cancer to allow intraoperative margin assessment.
Narrowband imaging uses filters to restrict to only two specific wavelengths in the visible light spectrum.
Using this to illuminate the tissues highlights the interpapillary capillary loop patterns in the mucosa, which are notably disorganised in tumour tissue \cite{vuEfficacyNarrowBand2014}.
Optical techniques as grouped here all measure the signal returned when light is shone on the target tissue.
Autofluorescence with the Velscope measures the photons emitted by fluorescence when monochromatic light is shone on the tissue.
Optical coherence tomography measures the refractive properties of tissue to construct a composite image of the tissue in 3D.
Raman spectroscopy quantifies the inelastic scattering of photons from tissue illuminated with monochromatic light.

Optical coherence tomography (OCT) uses changes in tissue optical properties---such as refractive index---to generate real-time, high-resolution cross-sectional images of tissue \cite{heidaria.e.UseOpticalCoherence2020}.

Autofluorescence techniques exploit the property of normal tissue to autofluoresce. 
It absorbs light and then emits photons at a longer wavelength \cite{leey.-j.IntraoperativeFluorescenceGuidedSurgery2020}.
Abnormal mucosa loses this property, and can thus be distinguished in real time.

\paragraph{Chemiluminescence}

A significant advance in recent years is the development of targeted chemiluminescent techniques, in which a tumour-specific marker is linked to a luminescent dye. 
This `smart' dye is then injected into the patient, and preferentially accumulates within the tumour. 

\subsection{Cutting edge IMA tools}

Here talk about future techniques, such as the iknife. 
This may end up being better in the discusssion.

% Objectives
Provide an explicit statement of questions being addressed with reference to participants, interventions, comparisons, outcomes, and study design (PICOS). 

This study aims to quantify the accuracy of tools and techniques used during radical surgical resection of head and neck squamous cell carcinoma to identify tissue types with the aim of identifying and correcting close or positive margins.


In the absence of sufficient studies with oncological outcomes to identify superior IMA techniques, we have gone for diagnostic outcome measures.
